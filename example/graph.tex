% ドキュメントの設定
\documentclass[a4paper,11pt,xelatex,ja=standard]{bxjsarticle}
\usepackage{tikz}
\usetikzlibrary {datavisualization.formats.functions}
\usepackage{pgfplots}
\usepackage{float}

% ドキュメント開始
\begin{document}


% 見出し
\section{\LaTeX のつかいかた}
% 小見出し
\subsection{hogex}
式\ref{eq:foobar}はfoobarの公式です。

% 数式
\begin{equation}
    \label{eq:foobar}
    foobar(x) = foo(x) + bar(x)
\end{equation}

\begin{tikzpicture}[scale=0.9]
    \datavisualization[ 
        scientific axes,
        visualize as line/.list={mydata, fit}, 
        mydata={style={thick,mark=*,red},label in legend={text=実験結果}},
        fit={style={thick,mark=*,blue},label in legend={text=実験結果2}},
        legend={north west inside},
        x axis={label={データ個数(個)},length=10cm},
        y axis={label={秒数(秒)},ticks={step=1.5},length=7cm},
     ]
     data[set=mydata] {
             x, y
             0, 0
         10000, 0.2
         20000, 0.6
         30000, 1.5
         40000, 2.0
         50000, 3.0
         60000, 3.5
         70000, 5.0
         80000, 6.0
         90000, 7.5
        100000, 9
    }
    data[set=fit] {
          x, y
         0, 0
         10000, 0.5
         20000, 1
         30000, 1.6
         40000, 2.2
         50000, 3.1
         60000, 3.8
         70000, 5.2
         80000, 6.4
         90000, 7.8
        100000, 9.5
    };
\end{tikzpicture}


\begin{figure}[H]
    \centering
    \begin{tikzpicture}[scale=0.9]
        \datavisualization[ 
            scientific axes,
            visualize as line/.list={index_a, index_b}, 
            index_a={style={thick,mark=*,black},label in legend={text=200lux}},
            index_b={style={thick,dashed,mark=triangle,black},label in legend={text=2万lux}},
            legend={north west outside},
            x axis={label={電圧[V]},length=10cm},
            y axis={label={電流[A]},length=6cm},
        ]
        data[set=index_a] {
            x, y
            0,0
        }
        data[set=index_b] {
            x, y
            0,0
        };
    \end{tikzpicture}
    \caption{I-V特性グラフ}
\end{figure}

\begin{figure}[H]
    \centering
    \begin{tikzpicture}[scale=0.9]
        \datavisualization[ 
            scientific axes,
            visualize as line/.list={index_a, index_b}, 
            index_a={style={thick,mark=*,black},label in legend={text=発電}},
            index_b={style={thick,dashed,mark=triangle,black},label in legend={text=充電}},
            legend={north west outside},
            x axis={label={風速[m/s]},length=10cm},
            y axis={label={電圧[V]},length=6cm},
        ]
        data[set=index_a] {
            x, y
            0,0
        }
        data[set=index_b] {
            x, y
            0,0
        };
    \end{tikzpicture}
    \caption{I-V特性グラフ}
\end{figure}

\end{document}

