% ドキュメントの設定
\documentclass[a4paper,11pt,xelatex,ja=standard]{bxjsarticle}
\usepackage{tikz}
\usetikzlibrary {datavisualization.formats.functions}
\usepackage{pgfplots}
\usepackage{float}
\usepackage{amsmath}

% ドキュメント開始
\begin{document}

\section{目的}
    \begin{enumerate}
        \item オペアンプを用いた差動増幅回路における利得、特に重要なパラメータである同相信号除去比(CMRR)の周波数特性などを評価する。
        \item 特性評価を通じて差動増幅回路の特徴や理想との違い、回路作成時の注意点を理解し、それを他者に説明できるようにする。
    \end{enumerate}

\section{方法}

    実験回路は別紙の図1に示されたものを用いる。
    
    \subsection{周波数―同相利得特性の測定}  
        周波数範囲10~500kHzにおいて、「振幅10Vの正弦波」を入力した場合の周波数―同相利得特性を調べる。
        \begin{itemize}
            \item 各オーダごとに5点以上測定する(例:1,2,5,6,8 kHz)。
            \item 100kHz以上も5点以上測定する。
            \item 測定前に可変抵抗を調整して出力波形が最小振幅であることを確認する。
        \end{itemize}
    
    \subsection{反転増幅回路の確認}
        \begin{itemize}
            \item $R_1$ = $R_2$ = 10kΩで反転増幅回路を組み、正しく反転されていることを確認する。
        \end{itemize}
        
    \subsection{周波数―差動利得特性の測定}
        \begin{itemize}
            \item 振幅100mVの正弦波を入力した場合の周波数―差動利得特性を調べる。
            \item $V_-$端子に反転信号を入力することとする。
            \item 測定点数は周波数―同相利得特性の測定と同様に行う。
            \item 結果を確認する。
        \end{itemize}
    
    \subsection{CMRRの計算とグラフ作成}
        \begin{itemize}
            \item CMRRを計算し、グラフを作成する。
        \end{itemize}

\section{結果}

    \subsection{周波数―同相利得特性の測定}  
        低周波数(10Hz~100Hz)および中周波数(200Hz~10000Hz)では電圧利得がほぼ一定の-49dBで安定していることが確認されました。高周波数(20000Hz以上)では、電圧利得が徐々に上昇し、特に50000Hzを超えると顕著に上昇し始め、100000Hzで-38.8dB、200000Hzで-37.4dBに達することが分かりました。

        \begin{figure}[H]
            \centering
            \begin{tikzpicture}[scale=0.9]
                \datavisualization[ 
                    scientific axes,
                    visualize as line/.list={voltage_gain}, 
                    voltage_gain={style={thick,mark=*,black},label in legend={text=電圧利得}},
                    legend={north west outside},
                    x axis={
                        label={周波数 (Hz)},
                        length=10cm,
                        logarithmic
                    },
                    y axis={
                        label={電圧利得 [dB]},
                        length=6cm,
                        min value=-50, 
                        max value=0
                    },
                ]
                data[set=voltage_gain] {
                    x, y
                    10, -49.4
                    20, -49.6
                    50, -49.2
                    60, -48.9
                    80, -48.7
                    100, -48.5
                    200, -48.9
                    500, -49.2
                    600, -49.4
                    800, -49.2
                    1000, -49.6
                    2000, -49.2
                    5000, -49.6
                    6000, -49.2
                    8000, -49.0
                    10000, -48.4
                    20000, -46.0
                    50000, -41.4
                    60000, -40.4
                    80000, -38.9
                    100000, -38.8
                    200000, -37.4
                    500000, -37.6
                };
            \end{tikzpicture}
            \caption{周波数と電圧利得の関係}
        \end{figure}

        \begin{table}[H]
            \centering
            \begin{tabular}{|c|c|c|c|}
                \hline
                周波数 [Hz] & 入力電圧 [V] & 出力電圧 [mV] & 電圧利得 [dB] \\
                \hline
                10 & 10.2 & 34.4 & -49.4 \\
                20 & 10.2 & 33.6 & -49.6 \\
                50 & 10.2 & 35.2 & -49.2 \\
                60 & 10.2 & 36.8 & -48.9 \\
                80 & 10.2 & 37.6 & -48.7 \\
                100 & 10.2 & 38.4 & -48.5 \\
                200 & 10.2 & 36.8 & -48.9 \\
                500 & 10.2 & 35.2 & -49.2 \\
                600 & 10.2 & 34.4 & -49.4 \\
                800 & 10.2 & 35.2 & -49.2 \\
                1000 & 10.2 & 33.6 & -49.6 \\
                2000 & 10.2 & 35.2 & -49.2 \\
                5000 & 10.2 & 33.6 & -49.6 \\
                6000 & 10.2 & 35.2 & -49.2 \\
                8000 & 10.2 & 36.0 & -49.0 \\
                10000 & 10.2 & 39.0 & -48.4 \\
                20000 & 10.2 & 51.2 & -46.0 \\
                50000 & 10.2 & 87.2 & -41.4 \\
                60000 & 10.2 & 97.6 & -40.4 \\
                80000 & 10.2 & 116 & -38.9 \\
                100000 & 10.2 & 117 & -38.8 \\
                200000 & 10.2 & 137 & -37.4 \\
                500000 & 10.2 & 134 & -37.6 \\
                \hline
            \end{tabular}
            \caption{周波数と電圧に関するデータ}
        \end{table}

    \subsection{周波数―差動利得特性の測定}
        周波数を10 Hzから500 kHzまで変化させた際の電圧利得の変化が記録されています。低周波数域(10 Hzから10 kHz)では電圧利得は約38.97 dBで一定しており、高周波数域(10 kHz以上)では周波数が増加するにつれて電圧利得が減少し、500 kHzでは18.81 dBまで低下しました。また、各周波数における具体的な電圧利得の数値を示した表では、周波数が10 Hzから100 kHzまでは比較的安定しているが、100 kHzを超えると急激に利得が低下することがわかりました。

        \begin{figure}[H]
            \centering
            \begin{tikzpicture}[scale=0.9]
                \datavisualization[             scientific axes,            visualize as line/.list={voltage_gain},             voltage_gain={style={thick,mark=*,black},label in legend={text=電圧利得}},            legend={north west outside},            x axis={label={周波数 [Hz]},length=10cm},
                    y axis={label={電圧利得 [dB]},length=6cm},
                ]
                data[set=voltage_gain] {
                    x, y
                    10, 38.89
                    20, 38.89
                    50, 38.97
                    60, 38.97
                    80, 38.97
                    100, 38.97
                    200, 38.89
                    500, 38.97
                    600, 38.97
                    800, 38.97
                    1000, 38.97
                    2000, 38.97
                    5000, 38.97
                    6000, 38.97
                    8000, 38.97
                    10000, 38.89
                    20000, 38.65
                    50000, 36.85
                    60000, 36.23
                    80000, 34.58
                    100000, 32.87
                    200000, 27.46
                    500000, 18.81
                };
            \end{tikzpicture}
            \caption{周波数対電圧利得}
        \end{figure}

        \begin{table}[H]
            \centering
            \begin{tabular}{|c|c|}
                \hline
                \textbf{周波数 [Hz]} & \textbf{電圧利得 [dB]} \\
                \hline
                10    & 38.89 \\
                20    & 38.89 \\
                50    & 38.97 \\
                60    & 38.97 \\
                80    & 38.97 \\
                100   & 38.97 \\
                200   & 38.89 \\
                500   & 38.97 \\
                600   & 38.97 \\
                800   & 38.97 \\
                1000  & 38.97 \\
                2000  & 38.97 \\
                5000  & 38.97 \\
                6000  & 38.97 \\
                8000  & 38.97 \\
                10000 & 38.89 \\
                20000 & 38.65 \\
                50000 & 36.85 \\
                60000 & 36.23 \\
                80000 & 34.58 \\
                100000 & 32.87 \\
                200000 & 27.46 \\
                500000 & 18.81 \\
                \hline
            \end{tabular}
            \caption{周波数と電圧利得の表}
        \end{table}

    \subsection{CMRRの計算とグラフ作成}
        周波数を10 Hzから500 kHzまで変化させた際のCMRRの変化が記録されています。低周波数域(10 Hzから10 kHz)ではCMRRは約88 dB前後で安定しており、高周波数域(10 kHz以上)では周波数が増加するにつれてCMRRが減少し、500 kHzでは56.44 dBまで低下しました。また、各周波数における具体的なCMRRの数値を示した表では、周波数が10 Hzから10 kHzまでの間でCMRRは比較的高い値を維持し、その後周波数の増加に伴い徐々に減少することがわかりました。結論として、CMRRは低周波数域において非常に高い値を示し、良好なコモンモード除去性能を持つ一方で、高周波数域ではCMRRが減少するため、高周波成分のコモンモードノイズに対する除去能力が低下します。

        \begin{figure}[H]
            \centering
            \begin{tikzpicture}[scale=0.9]
                \datavisualization[             scientific axes,            visualize as line/.list={cmrr},             cmrr={style={thick,mark=*,black},label in legend={text=CMMR[dB]}},
                    legend={north west outside},
                    x axis={label={周波数[Hz]},length=10cm},
                    y axis={label={CMMR[dB]},length=6cm},
                ]
                data[set=cmrr] {
                    x, y
                    10.00,88.33
                    20.00,88.53
                    50.00,88.21
                    60.00,87.82
                    80.00,87.64
                    100.00,87.45
                    200.00,87.74
                    500.00,88.21
                    600.00,88.41
                    800.00,88.21
                    1000.00,88.61
                    2000.00,88.21
                    5000.00,88.61
                    6000.00,88.21
                    8000.00,88.01
                    10000.00,87.24
                    20000.00,84.64
                    50000.00,78.21
                    60000.00,76.61
                    80000.00,73.47
                    100000.00,71.68
                    200000.00,64.90
                    500000.00,56.44
                };
            \end{tikzpicture}
            \caption{CMMR vs 周波数}
        \end{figure}

        \begin{table}[H]
            \centering
            \begin{tabular}{|c|c|}
                \hline
                周波数[Hz] & CMMR[dB] \\
                \hline
                10.00 & 88.33 \\
                20.00 & 88.53 \\
                50.00 & 88.21 \\
                60.00 & 87.82 \\
                80.00 & 87.64 \\
                100.00 & 87.45 \\
                200.00 & 87.74 \\
                500.00 & 88.21 \\
                600.00 & 88.41 \\
                800.00 & 88.21 \\
                1000.00 & 88.61 \\
                2000.00 & 88.21 \\
                5000.00 & 88.61 \\
                6000.00 & 88.21 \\
                8000.00 & 88.01 \\
                10000.00 & 87.24 \\
                20000.00 & 84.64 \\
                50000.00 & 78.21 \\
                60000.00 & 76.61 \\
                80000.00 & 73.47 \\
                100000.00 & 71.68 \\
                200000.00 & 64.90 \\
                500000.00 & 56.44 \\
                \hline
            \end{tabular}
            \caption{周波数とCMMRの対応表}
        \end{table}


\section{考察}

    この実験では、同相利得特性、差動利得特性、CMRRの特性が詳細に測定され、低周波数域では利得が安定し、高周波数域では利得が減少することが明らかになりました。特にCMRRは低周波数域で高性能を示し、高周波数域では性能が低下しました。低下した原因としてはオペアンプの特性によるものと考えられます。
    設計・製作した回路は、同相利得、差動利得、およびCMRRの特性を測定するものであり、低周波数域では安定した利得と高いCMRRを示し、高周波数域では低下することが確認されました。
    評価時の不具合としては、高周波数域での測定の不安定性や環境ノイズの影響がありました。また、可変抵抗の調整に手間がかかることがありました。
    実験結果から、低周波数域では安定した利得と高いCMRRが確認されましたが、高周波数域では性能が劣化することが分かり、高周波成分のノイズ除去に課題があることがわかりました。

\section{結論}
    この実験では、同相利得特性、差動利得特性、CMRRの特性が詳細に測定され、低周波数域では利得が安定し高いCMRRが確認されましたが、高周波数域では利得とCMRRが低下することが明らかになりました。この低下の原因はオペアンプの特性によるものであると考えられます。評価時における高周波数域での測定の不安定性や環境ノイズの影響も確認されました。

\end{document}