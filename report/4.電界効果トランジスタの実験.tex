% ドキュメントの設定
\documentclass[a4paper,11pt,xelatex,ja=standard]{bxjsarticle}
\usepackage{tikz}
\usetikzlibrary {datavisualization.formats.functions}
\usepackage{pgfplots}

% ドキュメント開始
\begin{document}

\section{実験の目的}

    トランジスタによって信号を適切に増幅するためにはバイアスの設定が重要となる。本実験では電界効果トランジスタのバイアス方法とソース接地増幅回路の基本特性を習得する。

\section{実験の理論または原理}
    \subsection{バイアス回路}
        トランジスタを動作させるために加える直流電圧を「バイアス電圧」といって、バイアスとして与える直流電流を「バイアス電流」と呼ぶ。適切に増幅された信号をトランジスタから得るためにはバイアスが必要になる。このバイアスを与える回路がバイアス回路であり、用途によっていくつかの回路が使い分けられている。
        \subsubsection{固定バイアス回路}
            固定バイアス回路では、ドレイン・ソース間電圧$V_{DS}$を与える電源$V_{DD}$に加えて、ゲート・ソース間電圧$V_{GS}$を与える電源$V_G$を別に設けて動作点を決定する。動作点はドレイン電流$I_D$の変化に対して変わらないため、電力増幅などのドレイン電流$I_D$が大きく変化する回路に用いられる。
        \subsubsection{自己バイアス回路}
            FETが小信号用として使われるときには、出力交流信号$d_i$の振幅がそれほど大きくないので、ドレイン電流$I_D$はほぼ一定とみなせる。そこで、ソースに直列に抵抗$R_S$を接続するとその電圧降下をバイアスとして利用することができる。これを自己バイアス回路と呼ぶ。自己バイアス回路では、何らかの原因でドレイン電流$I_D$が増加しても、ゲート・ソース間電圧$V_{GS}$が$I_D \times R_S$ によって決定されるため$(V_{GS} = - R_S I_D)$、$V_{GS}$がマイナス方向に増加して$I_D$の増加を抑える方向に働く。一方、$I_D$が減少すると$V_{GS}$が減少して、$I_D$の減少を抑える方向に働く。よって、自己バイアス回路は動作点の変動を防ぐように働くので小信号増幅回路では良く用いられる。一方、電力増幅回路のように$I_D$が大きく変化する回路に用いると、$I_D$の変化を抑えるように働くため不都合が生じてしまう。
        \subsubsection{固定バイアス法と自己バイアス法を併用する方式}
            このバイアス方法は、固定バイアス法と自己バイアス法を併用したもので、両者の長所を兼ね備えたバイアス法で、一般的に使用されている。
        \subsubsection{固定バイアス法と自己バイアス法を併用する方式}
            バイアス回路における電圧と電流の関係

            \begin{enumerate}
                \item[(a)] ゲート・ソース間電圧\( V_{GS} \)とドレイン電流\( I_D \)との関係
                \[ I_D = I_{DSS} \left( 1 - \frac{V_{GS}}{V_P} \right)^2 \quad (1) \]
            
                \item[(b)] ゲート電圧\( V_G \)と\( R_1 \)、\( R_2 \)(ブリーダ抵抗と呼ばれる)との関係
                \[ V_G = \frac{R_1}{R_1 + R_2} V_{DD} \quad (2) \]
            
                \item[(c)] ゲート・ソース間電圧\( V_{GS} \)とソース抵抗\( R_S \)の関係
                \[ V_{GS} = V_G - I_D R_S \quad (3) \]
                ゲート電圧に\( I_D R_S \)という電圧がフィードバックされ、安定度が改善される。
            
                \item[(d)] ソース抵抗\( R_S \)とドレイン電流\( I_D \)との関係
                \[ R_s = \frac{1 - \left( \frac{I_D}{I_{DSS}} \right) ^{\frac{1}{2}}}{I_D} \]

            \end{enumerate}
    \subsection{基本増幅回路}
        FET を使用する場合、その接地方式および出力の取り出し方により、次の3つに分類できる。
        \subsubsection{ソース接地増幅回路}
            ソース接地増幅回路は高入力インピーダンス、電圧利得を大きくできるといった特徴があり、良く用いられる回路である。
            \begin{enumerate}
                \item[(a)] 入力信号電圧$i_v$とすると、ドレイン電流の交流分$D_i$は次のように表される。
                    \[ i_D = g_m V_i \]
                \item[(b)] 出力電圧の交流分$o_v$は次式で表される。
                    \[ v_o = - i_D R_L = - g_m v_i R_L \]
                \item[(c)] 電圧増幅度$v_A$は次式で与えられる。
                    \[ A_v = \frac{V_o}{V_i} = - g_m R_L \]
            \end{enumerate}
        \subsubsection{ドレイン接地増幅回路(ソース・フォロア増幅回路)}
            ドレイン接地増幅回路は電圧利得が1以上になることはないが、高入力インピーダンスで入力信号を取り入れて、低出力インピーダンスで信号を送り出す、インピーダンス変換回路として用いられる。
        \subsubsection{ゲート接地増幅回路}
            この方式の場合には、電圧利得はソース接地の場合と同じであるが、入力インピーダンスを小さくできる。この場合、トランジスタでもよいが、高周波での安定な増幅に利点がある。
    
\section{実験の作業順序}
\section{実験の結果}
\section{実験の考察およびまとめ}
\section{参考文献}

\end{document}