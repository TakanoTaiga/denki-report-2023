% ドキュメントの設定
\documentclass[a4paper,11pt,xelatex,ja=standard]{bxjsarticle}
\usepackage{tikz}
\usetikzlibrary {datavisualization.formats.functions}
\usepackage{pgfplots}
\usepackage{float}

% ドキュメント開始
\begin{document}

\section{実験の目的}
    \begin{enumerate}
        \item トランジスタは基本的には,微小信号(電流)を増幅する能動素子の 1 つであり,これを用いて増幅回路を構成できるが,トランジスタを動作させるにはバイアス回路が必要である。本実験テーマでは,固定バイアス法と電流帰還形バイアス法の 2 種類についてエミッタ接地増幅回路を作製して,その増幅特性を測定し理解を深める。
        \item 演算増幅器(オペアンプ)はリニア IC の一つで,抵抗の接続方法により,反転増幅回路,非反転増幅回路を構成することができる。本実験テーマでは,オペアンプの動作について理解するとともに,反転,非反転増幅回路を作製し,その基本特性を理解する。
    \end
\section{実験の結果}
    \subsection{固定バイアス回路}
    \subsection{電流帰還バイアス回路}
\section{実験の考察およびまとめ}

\section{参考文献}

\end{document}