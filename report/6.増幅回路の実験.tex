% ドキュメントの設定
\documentclass[a4paper,11pt,xelatex,ja=standard]{bxjsarticle}
\usepackage{tikz}
\usetikzlibrary {datavisualization.formats.functions}
\usepackage{pgfplots}
\usepackage{float}

% ドキュメント開始
\begin{document}

\section{実験の目的}
    この実験の主題は、トランジスタと演算増幅器(オペアンプ)に焦点を当てています。トランジスタは微小信号を増幅するための素子であり、増幅回路を構築するためにはバイアス回路が必要です。実験では、固定バイアス法と電流帰還形バイアス法の2つの方法でエミッタ接地増幅回路を作成し、その増幅特性を測定して理解を深めます。また、演算増幅器(オペアンプ)は抵抗の接続方法により反転増幅回路と非反転増幅回路を構築できるリニアICの一種であり、実験ではオペアンプの動作を理解し、これらの増幅回路を作成して基本特性を理解します。
\section{実験の理論または原理}
    \subsection{増幅回路の接地方式による分類}

    接地方式には、ベース接地、エミッタ接地、コレクタ接地増幅回路の3種類があり、その特徴を下表に示す。最も多く用いられるのは、エミッタ接地である。

    \begin{table}[H]
    \centering
    \begin{tabular}{|c|c|c|c|}
    \hline
    項目 & ベース接地 & エミッタ接地 & コレクタ接地 \\
    \hline
    入力抵抗 $r_i$ & 小 & 中 & 大 \\
    出力抵抗 $r_o$ & 大 & 中 & 小 \\
    電流利得 $A_i$ & 小 ($\approx 1$) & 大 & 大 \\
    電圧利得 $A_v$ & 大 (負荷に比例) & 大 (負荷に比例) & 小 ($\approx 1$) \\
    電力利得 $G$ & 中 & 大 & 中 \\
    \hline
    \end{tabular}
    \caption{各接地方式と動作量}
    \end{table}

    \subsection{バイアス回路}

    一般にトランジスタやFETのような電子デバイスを動作させるには、その特性曲線の適当な位置に動作点(直流条件)を決める必要があります。動作点を与える直流電圧あるいは直流電流をそれぞれバイアス電圧(bias voltage)、バイアス電流(bias current)という。トランジスタは電流制御型なので、適当なバイアス電流を設定して動作点を決定する。

    エミッタ接地増幅回路ではバイアス電流としてベース電流 $I_B$ とコレクタ電流 $I_C$ を決める。通常、増幅回路ではひとつの直流電源を用いるので、このひとつの電源から2つのバイアス電流を設定しなければなりません。そこでバイアス回路が必要となります。バイアス回路については、電子回路の授業で詳しい内容を学習しているかもしれませんが、ここでは最も簡単な固定バイアス回路と、最も一般的に用いられる電流帰還バイアス回路について簡単に復習します。

    \subsection{トランジスタ増幅回路とhパラメータ}

    トランジスタ(以下、Tr)を用いて増幅回路を構成するとき、大きな電圧増幅度が得られるエミッタ接地増幅回路が最も多く用いられます。

    一方、Tr増幅回路の解析や設計を行う場合、Trの等価回路表現としてhパラメータによる等価回路表示が一般に用いられます。hパラメータの値はTrに与える直流条件(特にICの値)によって決まるため、hパラメータによる等価回路ではすでに動作点によって設定される直流条件が含まれているといってよい。図3にトランジスタのエミッタ接地とそれをhパラメータによる等価回路を示します。

    \begin{figure}[H]
    \centering
    \includegraphics[width=0.6\textwidth]{transistor_emitter_ground.png}
    \caption{トランジスタのエミッタ接地とhパラメータによる等価回路}
    \end{figure}

\section{実験の結果}
    \subsection{反転直流増幅回路}

        \begin{table}[htbp]
            \centering
            \caption{測定データ}
            \begin{tabular}{|c|c|c|c|c|}
            \hline
            $R2$ [$k\Omega$] & 理論値$A_v$ & 測定値$A_v$ & $E_o$ [V] & $E_i$ [V] \\
            \hline
            10.05 & -1.00 & -0.59 & -0.3 & 0.508 \\
            20.09 & -2.00 & -1.95 & -0.992 & 0.508 \\
            29.75 & -2.96 & -2.91 & -1.48 & 0.508 \\
            38.87 & -3.87 & -3.80 & -1.93 & 0.508 \\
            56.2  & -5.59 & -5.47 & -2.78 & 0.508 \\
            60.7  & -6.04 & -5.93 & -3.01 & 0.508 \\
            67.4  & -6.71 & -6.61 & -3.36 & 0.508 \\
            80.9  & -8.05 & -7.89 & -4.01 & 0.508 \\
            89.5  & -8.91 & -8.76 & -4.45 & 0.508 \\
            99.6  & -9.91 & -9.72 & -4.94 & 0.508 \\
            \hline
            \end{tabular}
            \label{tab:mytable}
        \end{table}

        \begin{figure}[H]
            \centering
            \begin{tikzpicture}
                \begin{axis}[
                    width=12cm,
                    height=8cm,
                    xlabel={$R2$ [$k\Omega$]},
                    ylabel={$A_v$},
                    legend style={at={(0.5,-0.15)},
                        anchor=north,legend columns=2},
                    symbolic x coords={10.05,20.09,29.75,38.87,56.2,60.7,67.4,80.9,89.5,99.6},
                    xtick=data,
                    ytick={-10,-9,-8,-7,-6,-5,-4,-3,-2,-1,0},
                    ymin=-10.5,
                    ymax=0.5,
                    grid=both,
                    ymajorgrids=true,
                    xticklabel style={rotate=45,anchor=east},
                    ]
                
                    \addplot[mark=*,thick] coordinates {
                        (10.05, -1.00)
                        (20.09, -2.00)
                        (29.75, -2.96)
                        (38.87, -3.87)
                        (56.2, -5.59)
                        (60.7, -6.04)
                        (67.4, -6.71)
                        (80.9, -8.05)
                        (89.5, -8.91)
                        (99.6, -9.91)
                    };
                    \addlegendentry{理論値$A_v$};
        
                    \addplot[mark=triangle,thick,dashed] coordinates {
                        (10.05, -0.59)
                        (20.09, -1.95)
                        (29.75, -2.91)
                        (38.87, -3.80)
                        (56.2, -5.47)
                        (60.7, -5.93)
                        (67.4, -6.61)
                        (80.9, -7.89)
                        (89.5, -8.76)
                        (99.6, -9.72)
                    };
                    \addlegendentry{測定値$A_v$};
                \end{axis}
            \end{tikzpicture}
            \caption{理論値と測定値の$A_v$ vs. $R2$ グラフ}
        \end{figure}
        
        \begin{table}[htbp]
            \centering
            \caption{測定データ}
            \begin{tabular}{|c|c|c|c|c|}
                \hline
                $R2$ [$k\Omega$] & 理論値$A_v$ & 測定値$A_v$ & $E_o$ [V] & $E_i$ [V] \\
                \hline
                56.2 & -5.59 & -3.34 & -13.3 & 3.98 \\
                56.2 & -5.59 & -3.80 & -13.3 & 3.5 \\
                56.2 & -5.59 & -4.38 & -13.3 & 3.04 \\
                56.2 & -5.59 & -5.34 & -13.3 & 2.49 \\
                56.2 & -5.59 & -5.56 & -11 & 1.98 \\
                56.2 & -5.59 & -5.54 & -8.36 & 1.51 \\
                56.2 & -5.59 & -5.57 & -5.54 & 0.994 \\
                56.2 & -5.59 & -5.57 & -2.79 & 0.501 \\
                56.2 & -5.59 & -5.69 & -1.12 & 0.197 \\
                56.2 & -5.59 & -5.61 & 1.15 & -0.205 \\
                56.2 & -5.59 & -5.63 & 2.82 & -0.501 \\
                56.2 & -5.59 & -5.64 & 5.64 & -1 \\
                56.2 & -5.59 & -5.61 & 8.42 & -1.5 \\
                56.2 & -5.59 & -5.63 & 11.2 & -1.99 \\
                56.2 & -5.59 & -5.58 & 13.9 & -2.49 \\
                56.2 & -5.59 & -4.70 & 14 & -2.98 \\
                56.2 & -5.59 & -3.99 & 14 & -3.51 \\
                56.2 & -5.59 & -3.51 & 14 & -3.99 \\
                \hline
            \end{tabular}
            \label{tab:mytable2}
        \end{table}

        \subsection{非反転直流増幅回路}

        \begin{table}[htbp]
            \centering
            \caption{新しい測定データ}
            \begin{tabular}{|c|c|c|c|c|}
                \hline
                $R2$ [$k\Omega$] & 理論値$A_v$ & 測定値$A_v$ & $E_o$ [V] & $E_i$ [V] \\
                \hline
                10.05 & 2.01 & 2.02 & 1.01 & 0.5 \\
                20.18 & 3.01 & 3.01 & 1.51 & 0.501 \\
                30.25 & 4.03 & 4.03 & 2.02 & 0.501 \\
                38.89 & 4.89 & 5.07 & 2.54 & 0.501 \\
                56.1  & 6.61 & 6.61 & 3.31 & 0.501 \\
                66.1  & 7.61 & 7.60 & 3.81 & 0.501 \\
                76.4  & 8.64 & 8.62 & 4.32 & 0.501 \\
                86.5  & 9.65 & 9.62 & 4.82 & 0.501 \\
                95.2  & 10.52 & 10.48 & 5.25 & 0.501 \\
                99.6  & 10.96 & 10.94 & 5.48 & 0.501 \\
                \hline
            \end{tabular}
            \label{tab:newdata}
        \end{table}
        \begin{figure}[H]
            \centering
            \begin{tikzpicture}
                \begin{axis}[
                    width=12cm,
                    height=8cm,
                    xlabel={$R2$ [$k\Omega$]},
                    ylabel={$A_v$},
                    legend style={at={(0.5,-0.15)},
                        anchor=north,legend columns=2},
                    symbolic x coords={10.05,20.18,30.25,38.89,56.1,66.1,76.4,86.5,95.2,99.6},
                    xtick=data,
                    ytick={0,1,2,3,4,5,6,7,8,9,10},
                    ymin=0,
                    ymax=11,
                    grid=both,
                    ymajorgrids=true,
                    xticklabel style={rotate=45,anchor=east},
                    ]
                
                    \addplot[mark=*,thick] coordinates {
                        (10.05, 2.01)
                        (20.18, 3.01)
                        (30.25, 4.03)
                        (38.89, 4.89)
                        (56.1, 6.61)
                        (66.1, 7.61)
                        (76.4, 8.64)
                        (86.5, 9.65)
                        (95.2, 10.52)
                        (99.6, 10.96)
                    };
                    \addlegendentry{理論値$A_v$};
        
                    \addplot[mark=triangle,thick,dashed] coordinates {
                        (10.05, 2.02)
                        (20.18, 3.01)
                        (30.25, 4.03)
                        (38.89, 5.07)
                        (56.1, 6.61)
                        (66.1, 7.60)
                        (76.4, 8.62)
                        (86.5, 9.62)
                        (95.2, 10.48)
                        (99.6, 10.94)
                    };
                    \addlegendentry{測定値$A_v$};
                \end{axis}
            \end{tikzpicture}
            \caption{新しい理論値と測定値の$A_v$ vs. $R2$ グラフ}
        \end{figure}
                
  
\section{実験の考察およびまとめ}
\section{参考文献}

\end{document}