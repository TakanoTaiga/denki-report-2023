% ドキュメントの設定
\documentclass[a4paper,11pt,xelatex,ja=standard]{bxjsarticle}
\usepackage{tikz}
\usetikzlibrary {datavisualization.formats.functions}
\usepackage{pgfplots}
\usepackage{float}

% ドキュメント開始
\begin{document}

\section{実験の目的}
    この実験の主題は、トランジスタと演算増幅器(オペアンプ)に焦点を当てています。トランジスタは微小信号を増幅するための素子であり、増幅回路を構築するためにはバイアス回路が必要です。実験では、固定バイアス法と電流帰還形バイアス法の2つの方法でエミッタ接地増幅回路を作成し、その増幅特性を測定して理解を深めます。また、演算増幅器(オペアンプ)は抵抗の接続方法により反転増幅回路と非反転増幅回路を構築できるリニアICの一種であり、実験ではオペアンプの動作を理解し、これらの増幅回路を作成して基本特性を理解します。

\end{document}