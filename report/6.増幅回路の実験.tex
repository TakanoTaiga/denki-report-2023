% ドキュメントの設定
\documentclass[a4paper,11pt,xelatex,ja=standard]{bxjsarticle}
\usepackage{tikz}
\usetikzlibrary {datavisualization.formats.functions}
\usepackage{pgfplots}
\usepackage{float}
\usepackage{amsmath}

% ドキュメント開始
\begin{document}

\section{実験の目的}
    \textbf{【1 週目】}トランジスタは基本的には,微小信号(電流)を増幅する能動素子の 1 つであり,これを用いて増幅回路を構成できるが,トランジスタを動作させるにはバイアス回路が必要である。本実験テーマでは,固定バイアス法と電流帰還形バイアス法の 2 種類についてエミッタ接地増幅回路を作製して,その増幅特性を測定し理解を深める。

    \textbf{【2 週目】}演算増幅器(オペアンプ)はリニア IC の一つで,抵抗の接続方法により,反転増幅回路,非反転増幅回路を構成することができる。本実験テーマでは,オペアンプの動作について理解するとともに,反転,非反転増幅回路を作製し,その基本特性を理解する。

\section{実験の理論または原理}
    \subsection{固定バイアス回路}
        固定バイアス回路におけるバイアス条件であるベース電流$I_B$とコレクタ・エミッタ間電圧$V_{CE}$を以下に示す。

        $$
        I_B = \frac{V_{CC} - V_{BE}}{R_B}
        $$

        $$
        I_C = \beta I_B
        $$

        $$
        V_{CE} = V_{CC} - I_C R_C = V_{CC} - \beta I_B R_C
        $$

    \subsection{電流帰還バイアス回路}

        $$
            I_1 = \frac{V_{CC} - V_{BE} - R_E I_E}{R_1} \quad (4)
        $$

        $$
            I_2 = \frac{V_{BE} + R_E I_E}{R_2} \quad (5)
        $$
        
        $$
            I_2 + I_B = I_1 \quad (6)
        $$

        これらの関係を用いて、$I_C$ を導くと次の式が得られる。

        $$
            I_C = \frac{\left(\frac{R_2}{R_1+R_2} \cdot V_{CC} - V_{BE}\right)}{{\left(\frac{{R_1 \cdot R_2}}{{R_1+R_2}} \cdot \frac{1}{\beta} + R_E\right)}}
        $$


\section{実験の結果と作業順序}
    \subsection{固定バイアス回路}
        固定バイアス回路において電源電圧 $V_{CC} = 12V$として$I_C = 2mA , V_{CE} = 6V$に設定したい。
        $T_R$の$\beta$の値を実測して以下の値を計算した。

        $$
        \beta = 156
        $$

        $$
        I_B = \frac{I_C}{\beta} = \frac{2 * 10^{-3}}{156} = 12.82 [uA]
        $$

        $$
        R_B = \frac{V_{CC} - V_{BE}}{I_B} = \frac{12 - 0.7}{12.82} = 882 [k \Omega]
        $$

        $$
        R_B = \frac{V_{CC} - V_{CE}}{I_C} = \frac{12 - 6}{2 * 10^{-3}} = 3 [k \Omega]
        $$

        求めた$R_B$と$R_C$の値から実際に使用する抵抗値に近いものを選択した。

        $$
        R_B = 820k + 51k = 881k \Omega
        $$

        $$
        R_B = 2.99k \Omega
        $$

        設定した抵抗を用いてブレッドボード上に回路を組む。そしてテスターを用いて設計値と比較する。

        $$
        I_C = \frac{V_{CC}-V_{CE}}{R_C} = \frac{12-5.97}{2.99} = 2.017 mA 
        $$

        $$
        V_{CE} = 6.02 V
        $$

    \subsection{電流帰還バイアス回路}
        電流帰還バイアス回路において電源電圧 $V_{CC} = 12V$として$I_C = 2mA , V_{CE} = 5V$に設定したい。
        $T_R$の$\beta$の値を実測して以下の値を計算した。

        $$
        R_E=\frac{V_{CC}-V_{CE}}{I_C} - R_C=\frac{12-5}{2*10^{-3}}-3*10^3= 500[\Omega]
        $$


        $$
         R_1 = \frac{V_{CC} - V_{BE} - R_E I_E}{I_1} = \frac{V_{CC} - V_{BE} - R_E I_C}{I_1} = \frac{12 - 0.7 - 500 \times 2 \times 10^{-3}}{0.2 \times 10^{-3}} = 51.5 \, \text{kΩ}
        $$

        \[ I_B = \frac{I_C}{\beta} = \frac{2 \times 10^{-3}}{156} = 12.8 \, \mu\text{A} \]


        \[ R_2 = \frac{V_{BE} + R_E I_E}{I_2} = \frac{0.7 + 500 \times 2 \times 10^{-3}}{0.2 \times 10^{-3} - 12.8 \times 10^{-6}} = 9.08 \, \text{kΩ} \]


        \[ \beta = 156 \]   \[ R_1 = 51.5 \, \text{kΩ} \]   \[ R_2 = 9.08 \, \text{kΩ} \]   \[ R_E = 500 \, \text{Ω} \]

        上記の実測値から以下の抵抗を設計した。


        \[ R_1 = 52 \, \text{kΩ} \, (\text{= 51 kΩ + 1 kΩ}) \] \[ R_2 = 9.1 \, \text{kΩ} \] \[ R_E = 500 \, \text{Ω} \]

        設計した抵抗値を用いて回路を組み$I_C$と$V_{CE}$の値をデジタルテスターを用いて計測し設計値と比較した。

        $$
        I_C = \frac{V_{RC}}{R_C} = \frac{5.82}{3.2k} = 1.82 mA
        $$

    \subsection{固定バイアスによるエミッタ接地増幅回路}
        固定バイアスによるエミッタ接地増幅回路で使用する抵抗値を以下の式を用いて求めた。

        $$
            R_B = \frac{V_{CC} - V_{BE}}{I_B} = \frac{12-0.7}{12.8 * 10^{-6}} = 881.4 k \Omega
        $$

        入力電圧$V_1 = 29.4[mV_{p-p}]$、周波数$f=1.04[kHz]$に設定して出力電圧を測定し電圧増幅度と電圧利得を求めた。
        
        出力電圧$V_2 = 5.00 [V_{p-p}]$

        電圧増幅度$A_v = \frac{V_2}{V_1} = \frac{5.00}{29.4} = 171.2[倍]$

        電圧利得$G_v = 20log|A_V| = 20log|171.2| = 44.7dB$ 

    \subsection{電流帰還バイアスによるエミッタ接地増幅回路}
        電流帰還バイアスによるエミッタ接地増幅回路において入力電圧$V1=29.5[mVp-p]$,周波数990Hzに設定して出力電圧を測定する。
        その値を用いて電圧増幅度と電圧利得を求めた。

        出力電圧 V2 = 3.01[Vp-p]
        電圧増幅度 Av = 104倍
        電圧利得 Gv = 20log|104| = 40.3 dB

        %実測値のβを用いてh_fe≒βとして近似し,また hie= 2.5 [kΩ] として電圧増幅度Av [倍] と電圧利得Gv [dB]の理論値を求め,測定値と比較した。 

        1. **電流帰還バイアス回路におけるICの導出**
        %電流帰還バイアス回路でのトランジスタのコレクタ電流ICを求めるため、与えられた式を操作します。I_EとI_Cはほぼ等しいとして、与えられた式に代入し、整理すると、ICは以下のように表されます。
        \[ IC = \frac{R_2 V_CC - (R_1 + R_2) V_BE}{R_1 R_E + R_2 R_E + \frac{R_1 R_2}{\beta}} \]

        2. **エミッタ接地増幅器の電圧増幅度Avの導出**
        エミッタ接地増幅器の電圧増幅度Avは、等価回路から導き出されます。入力電圧V1に対する出力電圧V2の比率として、Avは以下の式で表されます。
        \[ A_V = \frac{V_2}{V_1} = -\frac{h_{fe}}{h_{ie}} R_L \]

        3. **トランジスタ交換時のICの変化率の比較**
        固定バイアス回路と電流帰還バイアス回路で、トランジスタの特性のバラツキ(βが50増加)によるICの変化を比較します。固定バイアス回路ではICはβの変化に比例して約50増加しました。一方、電流帰還バイアス回路ではICはわずか3.4%の変化に留まりました。これにより、電流帰還バイアス回路がβのばらつきに対して安定していることが確認できます。

        4. **結合コンデンサと側路コンデンサの役割と影響**
        結合コンデンサは交流信号を通過させ、直流成分を遮断する役割を持ち、異なる動作電圧範囲を持つブロック間で交流信号を伝送する際に使用されます。結合コンデンサがない場合、直流成分が信号に乗ってしまい、信号が変質するリスクがあります。

        側路コンデンサは、エミッタ抵抗による電圧降下を交流信号に対して無効化する役割を果たします。これがないと、出力端子の動作電圧範囲が狭まり、増幅度が低下します。また、側路コンデンサがないと、特に低周波数での増幅度が大きく低下することが確認できます。

    \subsection{反転直流増幅回路の入出力特性の測定 }
        ブレッドボード上に反転直流増幅回路を作成する。R1 = R2 = R3 =10[kΩ]として回路を作成する。直流入力電圧を+0.6Vとして抵抗R2を変化しE0直流電圧増幅度を求めた。
        反転直流増幅回路において、仮想接地が理論だけでなく実験でも確認され、理論値計算にも活用可能であることがわかった。
        \begin{table}[htbp]
            \centering
            \caption{Your Table Caption}
            \label{tab:my-table}
            \begin{tabular}{|c|c|c|c|c|c|c|}
            \hline
            \textbf{R2[kΩ]} & \textbf{理論値Av} & \textbf{測定値Av} & \textbf{Eo[V]} & \textbf{Ei[V]} & \textbf{vcc+[V]} & \textbf{vcc-[-V]} \\ \hline
            10.05           & -1.00             & -0.59            & -0.3            & 0.508           & 15.04             & 15.03             \\ \hline
            20.09           & -2.00             & -1.95            & -0.992          & 0.508           &                   &                   \\ \hline
            29.75           & -2.96             & -2.91            & -1.48           & 0.508           &                   &                   \\ \hline
            38.87           & -3.87             & -3.80            & -1.93           & 0.508           &                   &                   \\ \hline
            56.2            & -5.59             & -5.47            & -2.78           & 0.508           &                   &                   \\ \hline
            60.7            & -6.04             & -5.93            & -3.01           & 0.508           &                   &                   \\ \hline
            67.4            & -6.71             & -6.61            & -3.36           & 0.508           & 15.04             & 15.03             \\ \hline
            80.9            & -8.05             & -7.89            & -4.01           & 0.508           &                   &                   \\ \hline
            89.5            & -8.91             & -8.76            & -4.45           & 0.508           &                   &                   \\ \hline
            99.6            & -9.91             & -9.72            & -4.94           & 0.508           &                   &                   \\ \hline
            \end{tabular}
            \end{table}
            
            \begin{figure}[H]
                \centering
                \begin{tikzpicture}
                    \begin{axis}[
                        xlabel={$R2[k\Omega]$},
                        ylabel={Voltage Gain ($A_v$)},
                        legend pos=north west,
                        grid=both,
                        grid style={line width=.1pt, draw=gray!10},
                        major grid style={line width=.2pt,draw=gray!50},
                        xmin=0, xmax=100,
                        ymin=-10, ymax=0,
                        xtick={0,10,20,30,40,50,60,70,80,90,100},
                        ytick={-10,-9,-8,-7,-6,-5,-4,-3,-2,-1,0},
                        ]
                        \addplot[only marks, mark=*] table {
                            R2    Av
                            10.05 -1.00
                            20.09 -2.00
                            29.75 -2.96
                            38.87 -3.87
                            56.2  -5.59
                            60.7  -6.04
                            67.4  -6.71
                            80.9  -8.05
                            89.5  -8.91
                            99.6  -9.91
                        };
                        \addplot[only marks, mark=triangle] table {
                            R2    Av
                            10.05 -0.59
                            20.09 -1.95
                            29.75 -2.91
                            38.87 -3.80
                            56.2  -5.47
                            60.7  -5.93
                            67.4  -6.61
                            80.9  -7.89
                            89.5  -8.76
                            99.6  -9.72
                        };
                        \legend{理論値Av, 測定値Av}
                    \end{axis}
                \end{tikzpicture}
                \caption{Voltage Gain vs. R2}
            \end{figure}
            

            ④	抵抗R2を 50 [kΩ] 程度に設定し,直流入力電圧Ei を-4 [V] ~+4 [V] まで変化させ,直
            流出力電圧Eoを測定する。
            直流入力電圧Eiが0から4[V]までの範囲では、直流電圧増幅度Avは比例的に増幅されるが、Ei=2.5[V]で電源±15[V]に飽和する。一方、0から-4[V]までの範囲では、Ei=-2.5[V]で飽和する。正と負のEiのバイアスを変えると、飽和するEiの値が異なる。
            \begin{table}[htbp]
                \centering
                \caption{Your table caption}
                \label{tab:my-table}
                \begin{tabular}{@{}cccccccc@{}}
                  $R2[\text{k}\Omega]$ & 理論値Av & 測定値Av & $Eo[\text{V}]$ & $Ei[\text{V}]$ & $v_{cc}+[\text{V}]$ & $v_{cc}-[\text{V}]$ \\ 
                  56.2 & -5.59 & -3.34 & -13.3 & 3.98 & 15.04 & 15.03 \\
                  56.2 & -5.59 & -3.80 & -13.3 & 3.5  &                     &                     \\
                  56.2 & -5.59 & -4.38 & -13.3 & 3.04 &                     &                     \\
                  56.2 & -5.59 & -5.34 & -13.3 & 2.49 &                     &                     \\
                  56.2 & -5.59 & -5.56 & -11   & 1.98 &                     &                     \\
                  56.2 & -5.59 & -5.54 & -8.36 & 1.51 &                     &                     \\
                  56.2 & -5.59 & -5.57 & -5.54 & 0.994 & 15.04 & 15.03 \\
                  56.2 & -5.59 & -5.57 & -2.79 & 0.501 &                     &                     \\
                  56.2 & -5.59 & -5.69 & -1.12 & 0.197 &                     &                     \\
                  56.2 & -5.59 & -5.61 & 1.15  & -0.205 &                     &                     \\
                  56.2 & -5.59 & -5.63 & 2.82  & -0.501 &                     &                     \\
                  56.2 & -5.59 & -5.64 & 5.64  & -1     &                     &                     \\
                  56.2 & -5.59 & -5.61 & 8.42  & -1.5   &                     &                     \\
                  56.2 & -5.59 & -5.63 & 11.2  & -1.99  &                     &                     \\
                  56.2 & -5.59 & -5.58 & 13.9  & -2.49  &                     &                     \\
                  56.2 & -5.59 & -4.70 & 14    & -2.98  &                     &                     \\
                  56.2 & -5.59 & -3.99 & 14    & -3.51  &                     &                     \\
                  56.2 & -5.59 & -3.51 & 14    & -3.99  &                     &                     \\ 
                \end{tabular}
              \end{table}

    
    \subsection{非反転直流増幅回路の入出力特性の測定}
    直流入力電圧を + 0.5[V] として,抵抗R2を 10~100 [kΩ]と変化させて直流出力電圧Eoを測定し,直流電圧増幅度Av を求める。また,測定値と理論値と比較する。  
    非反転直流増幅回路において、仮想接地が理論だけでなく実験でも確認され、理論値計算にも活用可能であることがわかった。

    \begin{table}[htbp]
        \centering
        \caption{Experimental Data}
        \label{tab:data}
        \begin{tabular}{cccccc}
        $R2 [\mathrm{k}\Omega]$ & 理論値 $A_v$ & 測定値 $A_v$ & $E_o [\mathrm{V}]$ & $E_i [\mathrm{V}]$ \\
        10.05 & 2.01 & 2.02 & 1.01 & 0.5 \\
        20.18 & 3.01 & 3.01 & 1.51 & 0.501 \\
        30.25 & 4.03 & 4.03 & 2.02 & 0.501 \\
        38.89 & 4.89 & 5.07 & 2.54 & 0.501 \\
        56.1 & 6.61 & 6.61 & 3.31 & 0.501 \\
        66.1 & 7.61 & 7.60 & 3.81 & 0.501 \\
        76.4 & 8.64 & 8.62 & 4.32 & 0.501 \\
        86.5 & 9.65 & 9.62 & 4.82 & 0.501 \\
        95.2 & 10.52 & 10.48 & 5.25 & 0.501 \\
        99.6 & 10.96 & 10.94 & 5.48 & 0.501 \\
            \end{tabular}
    \end{table}

    ④	抵抗R2を 50 [k] 程度に設定し,直流入力電圧Ei を-4 [V] ~+4 [V] まで変化させ,直流出力電圧Eoを測定する。

    
    \begin{table}[htbp]
        \centering
        \caption{実験データ}
          \begin{tabular}{cccccc}
          $R2[\mathrm{k}\Omega]$ & 理論値Av & 測定値Av & $Eo[\mathrm{V}]$ & $Ei[\mathrm{V}]$ \\
          56.2  & 6.62  & 3.48  & 13.9  & 3.99 \\
          56.2  & 6.59  & 3.97  & 13.9  & 3.50 \\
          56.2  & 6.62  & 4.62  & 13.9  & 3.01 \\
          56.2  & 6.62  & 5.52  & 13.9  & 2.52 \\
          56.2  & 6.62  & 6.57  & 13.2  & 2.01 \\
          56.2  & 6.62  & 6.59  & 9.88  & 1.50 \\
          56.2  & 6.62  & 6.59  & 6.56  & 1.00 \\
          56.2  & 6.62  & 6.61  & 3.29  & 0.50 \\
          56.2  & 6.62  & 6.47  & 1.3   & 0.20 \\
          56.2  & 6.62  & 6.56  & -1.28 & -0.20 \\
          56.2  & 6.62  & 6.56  & -3.19 & -0.49 \\
          56.2  & 6.62  & 6.62  & -6.62 & -1.00 \\
          56.2  & 6.62  & 6.60  & -9.83 & -1.49 \\
          56.2  & 6.62  & 6.60  & -13.2 & -2.00 \\
          56.2  & 6.62  & 5.32  & -13.3 & -2.50 \\
          56.2  & 6.62  & 4.43  & -13.3 & -3.00 \\
          56.2  & 6.62  & 3.80  & -13.3 & -3.50 \\
          56.2  & 6.62  & 3.33  & -13.3 & -3.99 \\
          \end{tabular}%
        \label{tab:data}%
      \end{table}%
      
    

\section{実験の考察およびまとめ}
\section{参考文献}

\end{document}