% ドキュメントの設定
\documentclass[a4paper,11pt,xelatex,ja=standard]{bxjsarticle}
\usepackage{tikz}
\usetikzlibrary {datavisualization.formats.functions}
\usepackage{pgfplots}
\usepackage{float}
\usepackage{amsmath}

% ドキュメント開始
\begin{document}

\section{実験の目的}

    私たちの生活は多様なエネルギー資源に依存しています。これらは一般的に「枯渇性エネルギー」(石炭、石油など)と「再生可能エネルギー」(太陽光、風力など)に大別されます。枯渇性エネルギーは使用に伴い減少し、再生可能エネルギーは使用しても資源が枯渇しないとされています。化石燃料の消費は二酸化炭素を発生させ、これは温室効果ガスとして地球温暖化を引き起こす可能性があります。したがって、持続可能なエネルギー利用と環境保護を両立させるためには、再生可能エネルギーの利用拡大が不可欠です。

\section{実験の理論または原理}
    \subsection{太陽電池の特性実験}
        太陽電池(またはフォトボルタイックセル、ソーラーセル)は、光起電力効果を活用して電力を生成するデバイスです。具体的には、シリコンなどの半導体材料に光を照射すると、材料内部で電荷が発生し、外部回路に接続することで電力を抽出することができます。太陽電池には様々なタイプが存在し、主要なものとして単結晶シリコン型、多結晶シリコン型、アモルファスシリコン型、および色素増感型があります。本実験では、単結晶および多結晶シリコン型の太陽電池を使用します。

        光起電力効果とは、光が半導体材料に入射すると、光のエネルギーが電子に転送され、これによって電子が励起されて電流が発生する現象を指します。太陽電池の作動はこの効果に基づいており、入射光に含まれるエネルギーが半導体内の電子を動かし、この電子の動き(電流)を利用して電力を得るのです。
    \subsection{風力発電の特性実験}
        風力発電は風の運動エネルギーを利用して電気エネルギーに変換する技術です。風のエネルギーは風車を回転させ、その回転エネルギーが発電機を動かし電力を生成します。風のエネルギーがどれだけ電力生成に寄与するかは、風速の3乗に比例します。これは、風の運動エネルギーが質量と速度の二乗に比例し、質量が風速に比例するため、結果的に風のエネルギーは風速の3乗に比例するからです。

        風車には主に2つのタイプがあり、水平軸型は風向に追尾する必要がありますが、大型化が可能です。一方、垂直軸型は風向に左右されないが、軸の長さと保持に課題があります。

\section{実験の作業順序}
    \subsection{太陽電池の特性実験}
        \subsubsection{開放電圧および短絡電流の照度依存性試験}
            使用機器:太陽電池特性実験装置、照度計

            実験手順:
            \begin{enumerate}
                \item 以下の設定に注意し、コンセントを電源に差し込みます。
                \item 太陽電池を引き抜き、照度計のセンサ部を太陽電池が置かれていた部分に挿入し、セレクトスイッチを「設定」に変更した後、スイッチをONにします。スライドトランスを調整して、照度が200 Lux になるように調整します。
                \item セレクトスイッチを「計測」に変更し、太陽電池を挿入してスイッチをONにします。そのときの発生電圧を記録します。
                \item 負荷抵抗を最小にして、太陽電池を挿入してスイッチをONにしたときの発電電流を記録します。
                \item 照度をパラメータにして繰り返します(200 Lux から 20,000 Lux まで)。
            \end{enumerate}

        \subsubsection{電流電圧(I-V)特性の実験}
            使用機器:太陽電池特性実験装置、照度計

            実験手順:
            \begin{enumerate}
                \item 太陽電池を引き抜き、照度計のセンサ部を太陽電池を置く部分に挿入し、セレクトスイッチを「設定」に変更した後、スイッチをONにします。スライドトランスを調整して、照度が20,000 Lux になるように調整します。
                \item 太陽電池を挿入し、セレクトスイッチを「計測」に変更します。そのとき、負荷抵抗を変化させたときの(100\% から 0\%)、発電電圧および発電電流を記録します。
                \item 上記を照度を2,000 Lux に変更して繰り返します。
            \end{enumerate}

        \subsubsection{太陽電池の直列および並列接続の実験}
            使用機器: 工作用太陽電池(2個)、照度計、デジタルマルチテスタ(2台)

            実験手順:
            \begin{enumerate}
                \item 太陽電池を引き抜き、照度計のセンサ部を太陽電池が置かれていた部分に挿入し、セレクトスイッチを「設定」に変更した後、スイッチをONにします。スライドトランスを調整して、照度が20,000 Lux になるように調整します。
                \item 工作用太陽電池(1つ)を受光面に置き、太陽電池に接続する抵抗を変化させながら、発電電圧と発電電流をデジタルマルチテスタで記録します。
                \item 工作用太陽電池を直列接続および並列接続にして、抵抗を変化させたときの発電電圧と発電電流をデジタルマルチテスタで記録します。
            \end{enumerate}


    \subsection{風力発電の特性実験}
        \subsubsection{風速と回転性能試験}
            結線: 制御盤の電圧・電流端子に電圧計および電流計を接続する。
    
            実験手順:
            \begin{enumerate}
                \item 電圧計・電流計を接続する。負荷スイッチを「負荷」にする。負荷抵抗を100 Ωにする。
                \item 「始動」ボタンを押す。
                \item 風速調整ボリュームを徐々に回し,周波数が5.0 Hzになるように設定したときの風速・回転数・発電電圧・消費電流を記録する。
                \item 周波数を5.0~50.0 Hzまで変更して,各種データを記録する。このとき,風車が回転する周波数・風速を記録すること。
                \item 風速調整ボリュームをゼロに戻し,「停止」ボタンを押した後,負荷抵抗を20 Ωに変更して,上記(2)~(5)の実験を繰り返す。
                \item 風速調整ボリュームをゼロに戻し,「停止」ボタンを押して停止させる。
            \end{enumerate}
    
        \subsubsection{風速と発電特性の実験}
            実験手順:
            \begin{enumerate}
                \item 「始動」ボタンを押す。
                \item 風速調整ボリュームを徐々に回し,周波数が35 Hzになるように設定したときの風速・回転数・発電電圧・消費電流を記録する。
                \item 抵抗値を変化させて(180~10 Ω),各種データを記録する。
                \item 周波数を50 Hzに変更して,上記(3)・(4)の実験を繰り返す。
                \item 風速調整ボリュームをゼロに戻し,「停止」ボタンを押して停止させる。
            \end{enumerate}
    
        \subsubsection{風速と充電特性の実験}
            結線: 制御盤内部アルミボックスに備え付けられた電圧・電流端子に電圧計および電流計を接続する。
    
            実験手順:
            \begin{enumerate}
                \item 負荷スイッチを「バッテリ」にする。
                \item 風速調整ボリュームを徐々に回し,周波数が5.0 Hzになるように設定したときの風速・回転数・発電電圧・消費電流・充電電圧・充電電流を記録する。
                \item 周波数を10.0~60.0 Hzまで変更して,各種データを記録する。このとき,風車が回転する周波数・風速を記録すること。
                \item 実験装置を停止させる。
                \item 周囲を清掃して実験を終了する。
            \end{enumerate}

\section{実験の結果}
    \subsection{太陽電池の特性実験}
        \subsubsection{開放電圧および短絡電流の照度依存性試験}

            このグラフは照度が0から約7,500 luxまでの範囲で増加すると、開放電圧も急速に増加することを示しています。照度が約2,500luxを超えると、開放電圧はほぼ一定の値(約18V)に飽和している。
            \begin{figure}[H]
                \centering
                \begin{tikzpicture}[scale=0.9]
                    \datavisualization[ 
                        scientific axes,
                        visualize as scatter/.list={voltage}, 
                        voltage={style={mark=*,black},label in legend={text=実験結果}},
                        legend={south east inside},
                        x axis={label={照度[lux]},length=10cm},
                        y axis={label={電圧[V]},ticks={step=2},length=6cm},
                    ]
                    data[set=voltage] {
                            x, y
                            7,	0.9
                            200,	16.7
                            500,	17.7
                            1000,	18.2
                            2000,	18.7
                            3500,	19
                            5000,	19.4
                            10000,	19.8
                            15000,	20.1
                            20000,	20.3
                    };
                \end{tikzpicture}
                \caption{開放電圧と照度の依存関係}
            \end{figure}

            \begin{table}[H]
                \centering
                \begin{tabular}{|c|c|}
                    \hline
                    \textbf{照度[lux]} & \textbf{電圧[V]} \\
                    \hline
                    7 & 0.9 \\
                    200 & 16.7 \\
                    500 & 17.7 \\
                    1000 & 18.2 \\
                    2000 & 18.7 \\
                    3500 & 19 \\
                    5000 & 19.4 \\
                    10000 & 19.8 \\
                    15000 & 20.1 \\
                    20000 & 20.3 \\
                    \hline
                \end{tabular}
                \caption{開放電圧と照度の依存関係}
            \end{table}
            

            グラフから、照度が増加するにつれて短絡電流も増加していることが確認できます。2,500luxから線形に推移していることがわかります。
            \begin{figure}[H]
                \centering
                \begin{tikzpicture}[scale=0.9]
                    \datavisualization[ 
                        scientific axes,
                        visualize as scatter/.list={amp}, 
                        amp={style={mark=*,black},label in legend={text=実験結果}},
                        legend={south east inside},
                        x axis={label={照度[lux]},length=10cm},
                        y axis={label={電流[A]},ticks={step=0.05},length=6cm},
                    ]
                    data[set=amp] {
                            x, y
                            7,	0
                            200,	0.012
                            500,	0.029
                            1000,	0.05
                            2000,	0.081
                            3500,	0.128
                            5000,	0.171
                            10000,	0.252
                            15000,	0.362
                            20000,	0.463
                    };
                \end{tikzpicture}
                \caption{短絡電流と照度の依存関係}
            \end{figure}

            \begin{table}[H]
                \centering
                \begin{tabular}{|c|c|}
                    \hline
                    \textbf{照度[lux]} & \textbf{電流[A]} \\
                    \hline
                    7 & 0 \\
                    200 & 0.012 \\
                    500 & 0.029 \\
                    1000 & 0.05 \\
                    2000 & 0.081 \\
                    3500 & 0.128 \\
                    5000 & 0.171 \\
                    10000 & 0.252 \\
                    15000 & 0.362 \\
                    20000 & 0.463 \\
                    \hline
                \end{tabular}
                \caption{短絡電流と照度の依存関係}
            \end{table}
            
        \subsubsection{電流電圧(I-V)特性の実験}
            このグラフは電圧(V)と電流(I)の関係を示すものです。15Vの電圧値まで電流は飽和しているが、その後急激に電流が減少することが確認できます。また2万luxと200luxを比較すると2万luxの方が電流が5倍程度多いのがわかる。
            \begin{figure}[H]
                \centering
                \begin{tikzpicture}[scale=0.9]
                    \datavisualization[ 
                        scientific axes,
                        visualize as scatter/.list={200lux, 20klux}, 
                        200lux={style={mark=*,black},label in legend={text=200lux}},
                        20klux={style={dashed,mark=triangle,black},label in legend={text=2万lux}},
                        legend={north west outside},
                        x axis={label={電圧[V]},length=10cm},
                        y axis={label={電流[A]},length=6cm},
                    ]
                    data[set=200lux] {
                        x, y
                        0,	    0.093
                        0.5,	0.095
                        3,	    0.094
                        4.5,	0.093
                        7.3,	0.093
                        8.9,	0.093
                        11.3,	0.093
                        13.4,	0.093
                        15.6,	0.092
                        16.3,	0.086
                        16.7,	0.082
                    }
                    data[set=20klux] {
                        x, y
                        1.4,	0.463
                        5.1,	0.462
                        8,	    0.454
                        14.6,	0.457
                        16.1,	0.443
                        18.2,	0.312
                        18.9,	0.237
                        19.2,	0.186
                        19.4,	0.15
                        19.5,	0.129
                        19.5,	0.113
                        19.6,	0.098
                        19.5,	0.095
                    };
                \end{tikzpicture}
                \caption{I-V特性グラフ}
            \end{figure}

            \begin{table}[H]
                \centering
                \caption{I-V特性データ}
                \begin{tabular}{|c|c|c|}
                    \hline
                    電圧[V] & 200luxの電流[A] & 2万luxの電流[A] \\
                    \hline
                    0.0 & 0.093 & - \\
                    0.5 & 0.095 & - \\
                    3.0 & 0.094 & - \\
                    4.5 & 0.093 & - \\
                    7.3 & 0.093 & - \\
                    8.9 & 0.093 & - \\
                    11.3 & 0.093 & - \\
                    13.4 & 0.093 & - \\
                    15.6 & 0.092 & - \\
                    16.3 & 0.086 & - \\
                    16.7 & 0.082 & - \\
                    1.4 & - & 0.463 \\
                    5.1 & - & 0.462 \\
                    8.0 & - & 0.454 \\
                    14.6 & - & 0.457 \\
                    16.1 & - & 0.443 \\
                    18.2 & - & 0.312 \\
                    18.9 & - & 0.237 \\
                    19.2 & - & 0.186 \\
                    19.4 & - & 0.150 \\
                    19.5 & - & 0.129 \\
                    19.5 & - & 0.113 \\
                    19.6 & - & 0.098 \\
                    19.5 & - & 0.095 \\
                    \hline
                \end{tabular}
            \end{table}
            

            このグラフは電力と電流の関係を示すものです。電力はある抵抗値のところまで上昇しその後減少することがこのグラフからわかる。
            \begin{figure}[H]
                \centering
                \begin{tikzpicture}[scale=0.9]
                    \datavisualization[ 
                        scientific axes,
                        visualize as scatter/.list={200lux, 20klux}, 
                        200lux={style={mark=*,black},label in legend={text=200lux}},
                        20klux={style={dashed,mark=triangle,black},label in legend={text=2万lux}},
                        legend={north west outside},
                        x axis={label={抵抗[Ω]},length=10cm},
                        y axis={label={電力[W]},length=6cm},
                    ]
                    data[set=200lux] {
                        x, y
                        0,	0
                        10,	0.0475
                        20,	0.282
                        30,	0.4185
                        40,	0.6789
                        50,	0.8277
                        60,	1.0509
                        70,	1.2462
                        80,	1.4352
                        90,	1.4018
                        100,	1.3694
                    }
                    data[set=20klux] {
                        x, y
                        0,	0.6482
                        10,	2.3562
                        15,	3.632
                        20,	6.6722
                        25,	7.1323
                        30,	5.6784
                        40,	4.4793
                        50,	3.5712
                        60,	2.91
                        70,	2.5155
                        80,	2.2035
                        90,	1.9208
                        100, 1.8525
                    };
                \end{tikzpicture}
                \caption{P-R特性グラフ}
            \end{figure}

            \begin{table}[H]
                \centering
                \caption{P-R特性データ}
                \begin{tabular}{|c|c|c|}
                    \hline
                    抵抗[Ω] & 200luxの電力[W] & 2万luxの電力[W] \\
                    \hline
                    0 & 0.000 & 0.6482 \\
                    10 & 0.0475 & 2.3562 \\
                    15 & - & 3.6320 \\
                    20 & 0.2820 & 6.6722 \\
                    25 & - & 7.1323 \\
                    30 & 0.4185 & 5.6784 \\
                    40 & 0.6789 & 4.4793 \\
                    50 & 0.8277 & 3.5712 \\
                    60 & 1.0509 & 2.9100 \\
                    70 & 1.2462 & 2.5155 \\
                    80 & 1.4352 & 2.2035 \\
                    90 & 1.4018 & 1.9208 \\
                    100 & 1.3694 & 1.8525 \\
                    \hline
                \end{tabular}
            \end{table}
            
        \subsubsection{太陽電池の直列および並列接続の実験}
            電圧が増加するにつれて電流も増加し、ある電圧でピークを迎えた後、電流が減少する特性を持っています。このピークは、それぞれの条件での最大電流を示しています。
            \begin{figure}[H]
                \centering
                \begin{tikzpicture}[scale=0.9]
                    \datavisualization[ 
                        scientific axes,
                        visualize as scatter/.list={ikko, heiretu, tyokuretu}, 
                        ikko={style={mark=*,black},label in legend={text=1個}},
                        heiretu={style={dashed,mark=triangle,black},label in legend={text=並列}},
                        tyokuretu={style={dotted,mark=diamond,black},label in legend={text=直列}},
                        legend={north west outside},
                        x axis={label={電圧[V]},length=10cm},
                        y axis={label={電流[A]},length=6cm},
                    ]
                    data[set=ikko] {
                        x, y
                        1.086,	217.8
                        1.099,	210.6
                        1.161,	210.6
                        1.256,	209.1
                        1.439,	204.6
                        1.582,	158.3
                        1.675,	111.8
                        1.708,	68.5
                        1.713,	49.2
                        1.716,	38.3
                        1.714,	31.3
                        1.716,	26.5
                        1.713,	23
                        1.709,	20.2
                        1.704,	18
                        1.702,	16.3
                    }
                    data[set=heiretu] {
                        x, y
                        0.304,	1.099
                        1.606,	290.7
                        1.625,	271
                        1.664,	236.8
                        1.697,	169.9
                        1.706,	114
                        1.712,	68.8
                        1.712,	49.1
                        1.709,	38.2
                        1.706,	31.1
                        1.706,	26.4
                        1.701,	22.8
                        1.697,	20
                        1.691,	17.9
                        1.688,	16.1
                    }
                    data[set=tyokuretu] {
                        x, y
                        1.09,	217.8
                        1.188,	230.3
                        1.261,	229
                        1.361,	225.9
                        1.52,	216.6
                        2.16,	217
                        2.92,	194.5
                        3.19,	128.2
                        3.26,	93.8
                        3.31,	74
                        3.34,	61.1
                        3.36,	52
                        3.38,	45.4
                        3.4,	40.3
                        3.42,	36.2
                        3.41,	32.7
                    };
                \end{tikzpicture}
                \caption{I-V特性グラフ}
            \end{figure}


            \begin{table}[H]
                \centering
                \caption{I-V特性データ}
                \begin{tabular}{|c|c|c|c|}
                    \hline
                    電圧[V] & 1個の電流[A] & 並列の電流[A] & 直列の電流[A] \\
                    \hline
                    1.086 & 217.8 & - & 217.8 \\
                    1.099 & 210.6 & - & 230.3 \\
                    1.161 & 210.6 & - & 229.0 \\
                    1.256 & 209.1 & - & 225.9 \\
                    1.439 & 204.6 & - & 216.6 \\
                    1.582 & 158.3 & - & 217.0 \\
                    1.675 & 111.8 & - & 194.5 \\
                    1.708 & 68.5 & - & 128.2 \\
                    1.713 & 49.2 & - & 93.8 \\
                    1.716 & 38.3 & - & 74.0 \\
                    1.714 & 31.3 & - & 61.1 \\
                    1.716 & 26.5 & - & 52.0 \\
                    1.713 & 23.0 & - & 45.4 \\
                    1.709 & 20.2 & - & 40.3 \\
                    1.704 & 18.0 & - & 36.2 \\
                    1.702 & 16.3 & - & 32.7 \\
                    0.304 & - & 1.099 & - \\
                    1.606 & - & 290.7 & - \\
                    1.625 & - & 271.0 & - \\
                    1.664 & - & 236.8 & - \\
                    1.697 & - & 169.9 & - \\
                    1.706 & - & 114.0 & - \\
                    1.712 & - & 68.8 & - \\
                    1.712 & - & 49.1 & - \\
                    1.709 & - & 38.2 & - \\
                    1.706 & - & 31.1 & - \\
                    1.706 & - & 26.4 & - \\
                    1.701 & - & 22.8 & - \\
                    1.697 & - & 20.0 & - \\
                    1.691 & - & 17.9 & - \\
                    1.688 & - & 16.1 & - \\
                    1.09 & - & - & 217.8 \\
                    1.188 & - & - & 230.3 \\
                    1.261 & - & - & 229.0 \\
                    1.361 & - & - & 225.9 \\
                    1.52 & - & - & 216.6 \\
                    2.16 & - & - & 217.0 \\
                    2.92 & - & - & 194.5 \\
                    3.19 & - & - & 128.2 \\
                    3.26 & - & - & 93.8 \\
                    3.31 & - & - & 74.0 \\
                    3.34 & - & - & 61.1 \\
                    3.36 & - & - & 52.0 \\
                    3.38 & - & - & 45.4 \\
                    3.4 & - & - & 40.3 \\
                    3.42 & - & - & 36.2 \\
                    3.41 & - & - & 32.7 \\
                    \hline
                \end{tabular}
            \end{table}

            

            抵抗が増加するにつれて電力が減少する傾向が示されています。このグラフは抵抗と電力の関係を示しています。

            \begin{figure}[H]
                \centering
                \begin{tikzpicture}[scale=0.9]
                    \datavisualization[ 
                        scientific axes,
                        visualize as scatter/.list={ikko, heiretu, tyokuretu}, 
                        ikko={style={mark=*,black},label in legend={text=1個}},
                        heiretu={style={dashed,mark=triangle,black},label in legend={text=並列}},
                        tyokuretu={style={dotted,mark=diamond,black},label in legend={text=直列}},
                        legend={north west outside},
                        x axis={label={抵抗[Ω]},length=10cm},
                        y axis={label={電力[W]},length=6cm},
                    ]
                    data[set=ikko] {
                        x, y
                        0,	    0.2365308
                        0.4,	231.4494
                        1,	244.5066
                        2,	    262.6296
                        4,	    294.4194
                        10,	    250.4306
                        20, 	187.265
                        40, 	116.998
                        60,	    84.2796
                        80,	    65.7228
                        100,	    53.6482
                        120,	    45.474
                        140,	    39.399
                        160, 	34.5218
                        180, 	30.672
                        200,	27.7426
                    }
                    data[set=heiretu] {
                        x, y
                        0.4,	334.4257
                        1,	466.8642
                        2,	    440.375
                        4,	    394.0352
                        10,	    288.3203
                        20, 	194.484
                        40, 	117.7856
                        60,	    84.0592
                        80,	    65.2838
                        100,	    53.0566
                        120,	    45.0384
                        140, 	38.7828
                        160, 	33.94
                        180, 	30.2689
                        200,	27.1768
                    }
                    data[set=tyokuretu] {
                        x, y
                        0,	    0.237402
                        0.4,	273.5964
                        1,	288.769
                        2,	    307.4499
                        4,	    329.232
                        10,	    468.72
                        20,	    567.94
                        40,	    408.958
                        60,	    305.788
                        80, 	244.94
                        100,	    204.074
                        120,	    174.72
                        140,	    153.452
                        160,	    137.02
                        180,	    123.804
                        200,	111.507
                    };
                \end{tikzpicture}
                \caption{P-R特性グラフ}
            \end{figure}


            \begin{table}[H]
                \centering
                \caption{抵抗と電力の関係}
                \begin{tabular}{|c|c|c|c|}
                    \hline
                    抵抗[Ω] & 1個の電力[W] & 並列の電力[W] & 直列の電力[W] \\
                    \hline
                    0 & 0.2365308 & 0.237402 & 0.2365308 \\
                    0.4 & 231.4494 & 334.4257 & 273.5964 \\
                    1 & 244.5066 & 466.8642 & 288.769 \\
                    2 & 262.6296 & 440.375 & 307.4499 \\
                    4 & 294.4194 & 394.0352 & 329.232 \\
                    10 & 250.4306 & 288.3203 & 468.72 \\
                    20 & 187.265 & 194.484 & 567.94 \\
                    40 & 116.998 & 117.7856 & 408.958 \\
                    60 & 84.2796 & 84.0592 & 305.788 \\
                    80 & 65.7228 & 65.2838 & 244.94 \\
                    100 & 53.6482 & 53.0566 & 204.074 \\
                    120 & 45.474 & 45.0384 & 174.72 \\
                    140 & 39.399 & 38.7828 & 153.452 \\
                    160 & 34.5218 & 33.94 & 137.02 \\
                    180 & 30.672 & 30.2689 & 123.804 \\
                    200 & 27.7426 & 27.1768 & 111.507 \\
                    \hline
                \end{tabular}
            \end{table}
            

    \subsection{風力発電の特性実験}
        \subsubsection{風速と回転性能試験}

            このグラフは風速と電圧の関係を表しています。負荷抵抗に関わらず風速に対して電圧が線形に推移しています。

            \begin{figure}[H]
                \centering
                \begin{tikzpicture}[scale=0.9]
                    \datavisualization[ 
                        scientific axes,
                        visualize as scatter/.list={index_a, index_b}, 
                        index_a={style={mark=*,black},label in legend={text=負荷抵抗:100Ω}},
                        index_b={style={dashed,mark=triangle,black},label in legend={text=負荷抵抗:20Ω}},
                        legend={north west inside},
                        x axis={label={風速[m/s]},length=10cm},
                        y axis={label={電圧[V]},length=5cm},
                    ]
                    data[set=index_a] {
                        x, y
                        1.3,	0
                        2.3,	3.5
                        3.4,	6.4
                        4.6,	10
                        5.6,	13.2
                        7.0,	17.2
                        7.8,	20.8
                        9.2,	24.6
                        10.2,	28.4
                    }
                    data[set=index_b] {
                        x, y
                        1.3,	0
                        2.3,	1.3
                        3.3,	2.2
                        4.5,	3.6
                        5.7,	5.2
                        6.7,	7
                        8.0,	9.1
                        9.2,	10.1
                        10.3,	12.9
                    };
                \end{tikzpicture}
                \caption{風速と電圧の特性グラフ}
            \end{figure}

            \begin{table}[H]
                \centering
                \caption{風速と電圧の特性データ}
                \begin{tabular}{|c|c|c|}
                    \hline
                    風速[m/s] & 負荷抵抗:100Ωの電圧[V] & 負荷抵抗:20Ωの電圧[V] \\
                    \hline
                    1.3 & 0 & 0 \\
                    2.3 & 3.5 & 1.3 \\
                    3.4 & 6.4 & 2.2 \\
                    4.6 & 10 & 3.6 \\
                    5.6 & 13.2 & 5.2 \\
                    7.0 & 17.2 & 7 \\
                    7.8 & 20.8 & 9.1 \\
                    9.2 & 24.6 & 10.1 \\
                    10.2 & 28.4 & 12.9 \\
                    \hline
                \end{tabular}
            \end{table}
            

            このグラフは風速と電流の関係を表しています。負荷抵抗に関わらず風速に対して電流が線形に推移しています。

            \begin{figure}[H]
                \centering
                \begin{tikzpicture}[scale=0.9]
                    \datavisualization[ 
                        scientific axes,
                        visualize as scatter/.list={index_a, index_b}, 
                        index_a={style={mark=*,black},label in legend={text=負荷抵抗:100Ω}},
                        index_b={style={dashed,mark=triangle,black},label in legend={text=負荷抵抗:20Ω}},
                        legend={north west inside},
                        x axis={label={風速[m/s]},length=10cm},
                        y axis={label={電流[A]},length=6cm},
                    ]
                    data[set=index_a] {
                        x, y
                        1.3,	0
                        2.3,	0.03
                        3.4,	0.06
                        4.6,	0.1
                        5.6,	0.13
                        7.0,	0.18
                        7.8,	0.22
                        9.2,	0.26
                        10.2,	0.3
                    }
                    data[set=index_b] {
                        x, y
                        1.3,	0
                        2.3,	0.08
                        3.3,	0.2
                        4.5,	0.35
                        5.7,	0.52
                        6.7,	0.7
                        8.0,	0.94
                        9.2,	1.14
                        10.3,	1.32
                    };
                \end{tikzpicture}
                \caption{風速と電流特性グラフ}
            \end{figure}

            \begin{table}[H]
                \centering
                \caption{風速と電流の特性データ}
                \begin{tabular}{|c|c|c|}
                    \hline
                    風速[m/s] & 負荷抵抗:100Ωの電流[A] & 負荷抵抗:20Ωの電流[A] \\
                    \hline
                    1.3 & 0 & 0 \\
                    2.3 & 0.03 & 0.08 \\
                    3.4 & 0.06 & 0.2 \\
                    4.6 & 0.1 & 0.35 \\
                    5.6 & 0.13 & 0.52 \\
                    7.0 & 0.18 & 0.7 \\
                    7.8 & 0.22 & 0.94 \\
                    9.2 & 0.26 & 1.14 \\
                    10.2 & 0.3 & 1.32 \\
                    \hline
                \end{tabular}
            \end{table}
            

            このグラフは風速と電力の関係を表しています。負荷抵抗に関わらず風速に対して電力が$X^n$の形になることがわかります。

            \begin{figure}[H]
                \centering
                \begin{tikzpicture}[scale=0.9]
                    \datavisualization[ 
                        scientific axes,
                        visualize as scatter/.list={index_a, index_b}, 
                        index_a={style={mark=*,black},label in legend={text=負荷抵抗:100Ω}},
                        index_b={style={dashed,mark=triangle,black},label in legend={text=負荷抵抗:20Ω}},
                        legend={north west inside},
                        x axis={label={風速[m/s]},length=10cm},
                        y axis={label={電力[W]},length=6cm},
                    ]
                    data[set=index_a] {
                        x, y
                        1.3,	0
                        2.3,	0.105
                        3.4,	0.384
                        4.6,	1
                        5.6,	1.716
                        7.0,	3.096
                        7.8,	4.576
                        9.2,	6.396
                        10.2,	8.52
                    }
                    data[set=index_b] {
                        x, y
                        1.3,	0
                        2.3,	0.104
                        3.3,	0.44
                        4.5,	1.26
                        5.7,	2.704
                        6.7,	4.9
                        8.0,	8.554
                        9.2,	11.514
                        10.3,	17.028
                    };
                \end{tikzpicture}
                \caption{風速と電力の特性グラフ}
            \end{figure}

            \begin{table}[H]
                \centering
                \caption{風速と電力の特性データ}
                \begin{tabular}{|c|c|c|}
                    \hline
                    風速[m/s] & 負荷抵抗:100Ωの電力[W] & 負荷抵抗:20Ωの電力[W] \\
                    \hline
                    1.3 & 0 & 0 \\
                    2.3 & 0.105 & 0.104 \\
                    3.4 & 0.384 & 0.44 \\
                    4.6 & 1 & 1.26 \\
                    5.6 & 1.716 & 2.704 \\
                    7.0 & 3.096 & 4.9 \\
                    7.8 & 4.576 & 8.554 \\
                    9.2 & 6.396 & 11.514 \\
                    10.2 & 8.52 & 17.028 \\
                    \hline
                \end{tabular}
            \end{table}
            

        \subsubsection{風速と発電特性の実験}

            このグラフは負荷抵抗と発電力の関係を関係を風の強さ別に表したグラフです。30$\%$付近で電力が最大となりその後減少していきます。

            \begin{figure}[H]
                \centering
                \begin{tikzpicture}[scale=0.9]
                    \datavisualization[ 
                        scientific axes,
                        visualize as scatter/.list={index_a, index_b}, 
                        index_a={style={mark=*,black},label in legend={text=35Hz}},
                        index_b={style={dashed,mark=triangle,black},label in legend={text=50Hz}},
                        legend={north east inside},
                        x axis={label={負荷抵抗[Ω]},length=10cm},
                        y axis={label={電力[W]},length=6cm},
                    ]
                    data[set=index_a] {
                        x, y
                        10,	    0.62
                        20,	    3.828
                        30, 	8.364
                        40, 	8.236
                        50, 	7.824
                        60, 	6.65
                        70, 	6.072
                        80,	    5.376
                        90, 	5.025
                        100,	4.51
                        120,	3.638
                        140,	3.27
                        160,	2.873
                        180,	2.676
                    }
                    data[set=index_b] {
                        x, y
                        10, 	2.272
                        20, 	12.342
                        30, 	24.708
                        40, 	22.442
                        50, 	19.24
                        60, 	17.05
                        70, 	14.82
                        80, 	13.41
                        90,	    11.895
                        100,	10.71
                        120,	9.1
                        140,	7.59
                        160,	7.035
                        180,	6.192
                    };
                \end{tikzpicture}
                \caption{風速と発電特性の特性グラフ}
            \end{figure}

            \begin{table}[H]
                \centering
                \caption{負荷抵抗と電力の特性データ}
                \begin{tabular}{|c|c|c|}
                    \hline
                    負荷抵抗[Ω] & 35Hzの電力[W] & 50Hzの電力[W] \\
                    \hline
                    10 & 0.62 & 2.272 \\
                    20 & 3.828 & 12.342 \\
                    30 & 8.364 & 24.708 \\
                    40 & 8.236 & 22.442 \\
                    50 & 7.824 & 19.24 \\
                    60 & 6.65 & 17.05 \\
                    70 & 6.072 & 14.82 \\
                    80 & 5.376 & 13.41 \\
                    90 & 5.025 & 11.895 \\
                    100 & 4.51 & 10.71 \\
                    120 & 3.638 & 9.1 \\
                    140 & 3.27 & 7.59 \\
                    160 & 2.873 & 7.035 \\
                    180 & 2.676 & 6.192 \\
                    \hline
                \end{tabular}
            \end{table}
            

        \subsubsection{風速と充電特性の実験}

            発電電圧と充電電圧の特性を表したグラフです。

            \begin{figure}[H]
                \centering
                \begin{tikzpicture}[scale=0.9]
                    \datavisualization[ 
                        scientific axes,
                        visualize as scatter/.list={index_a, index_b}, 
                        index_a={style={mark=*,black},label in legend={text=発電}},
                        index_b={style={dashed,mark=triangle,black},label in legend={text=充電}},
                        legend={north west outside},
                        x axis={label={風速[m/s]},length=10cm},
                        y axis={label={電圧[V]},length=6cm},
                    ]
                    data[set=index_a] {
                        x, y
                        1.3,	0
                        2.2,	4.89
                        4.5,	11.64
                        6.6,	12.09
                        9.2,	12.41
                        11.2,	12.76
                        13.6,	13.24
                    }
                    data[set=index_b] {
                        x, y
                        1.3,	11.5
                        2.2,	11.5
                        4.5,	11.6
                        6.6,	11.6
                        9.2,	11.7
                        11.2,	11.9
                        13.6,	12.2
                    };
                \end{tikzpicture}
                \caption{発電電圧と充電電圧の特性グラフ}
            \end{figure}

            \begin{table}[H]
                \centering
                \caption{発電電圧と充電電圧の特性データ}
                \begin{tabular}{|c|c|c|}
                    \hline
                    風速[m/s] & 発電電圧[V] & 充電電圧[V] \\
                    \hline
                    1.3 & 0 & 11.5 \\
                    2.2 & 4.89 & 11.5 \\
                    4.5 & 11.64 & 11.6 \\
                    6.6 & 12.09 & 11.6 \\
                    9.2 & 12.41 & 11.7 \\
                    11.2 & 12.76 & 11.9 \\
                    13.6 & 13.24 & 12.2 \\
                    \hline
                \end{tabular}
            \end{table}

            
            発電電流と充電電流の特性を表したグラフです。

            \begin{figure}[H]
                \centering
                \begin{tikzpicture}[scale=0.9]
                    \datavisualization[ 
                        scientific axes,
                        visualize as scatter/.list={index_a, index_b}, 
                        index_a={style={mark=*,black},label in legend={text=発電}},
                        index_b={style={dashed,mark=triangle,black},label in legend={text=充電}},
                        legend={south east inside},
                        x axis={label={風速[m/s]},length=10cm},
                        y axis={label={電流[A]},length=6cm},
                    ]
                    data[set=index_a] {
                        x, y
                        1.3,	0
                        2.2,	0
                        4.5,	0.03
                        6.6,	0.4
                        9.2,	0.93
                        11.2,	1.64
                        13.6,	2.41
                    }
                    data[set=index_b] {
                        x, y
                        1.3,	0
                        2.2,	0
                        4.5,	0
                        6.6,	0.4
                        9.2,	0.99
                        11.2,	1.72
                        13.6,	2.58
                    };
                \end{tikzpicture}
                \caption{発電電流と充電電流の特性グラフ}
            \end{figure}

            \begin{table}[H]
                \centering
                \caption{発電電流と充電電流の特性データ}
                \begin{tabular}{|c|c|c|}
                    \hline
                    風速[m/s] & 発電電流[A] & 充電電流[A] \\
                    \hline
                    1.3 & 0 & 0 \\
                    2.2 & 0 & 0 \\
                    4.5 & 0.03 & 0 \\
                    6.6 & 0.4 & 0.4 \\
                    9.2 & 0.93 & 0.99 \\
                    11.2 & 1.64 & 1.72 \\
                    13.6 & 2.41 & 2.58 \\
                    \hline
                \end{tabular}
            \end{table}            

            発電電力と充電電力の特性を表したグラフです。

            \begin{figure}[H]
                \centering
                \begin{tikzpicture}[scale=0.9]
                    \datavisualization[ 
                        scientific axes,
                        visualize as scatter/.list={index_a, index_b}, 
                        index_a={style={mark=*,black},label in legend={text=発電}},
                        index_b={style={dashed,mark=triangle,black},label in legend={text=充電}},
                        legend={north west inside},
                        x axis={label={風速[m/s]},length=10cm},
                        y axis={label={電力[A]},length=6cm},
                    ]
                    data[set=index_a] {
                        x, y
                        1.3,	0
                        2.2,	0
                        4.5,	0.3492
                        6.6,	4.836
                        9.2,	11.5413
                        11.2,	20.9264
                        13.6,	31.9084
                    }
                    data[set=index_b] {
                        x, y
                        1.3,	0
                        2.2,	0
                        4.5,	0
                        6.6,	4.64
                        9.2,	11.583
                        11.2,	20.468
                        13.6,	31.476
                    };
                \end{tikzpicture}
                \caption{発電電力と充電電力の特性グラフ}
            \end{figure}

            \begin{table}[H]
                \centering
                \caption{発電電力と充電電力の特性データ}
                \begin{tabular}{|c|c|c|}
                    \hline
                    風速[m/s] & 発電電力[A] & 充電電力[A] \\
                    \hline
                    1.3 & 0 & 0 \\
                    2.2 & 0 & 0 \\
                    4.5 & 0.3492 & 0 \\
                    6.6 & 4.836 & 4.64 \\
                    9.2 & 11.5413 & 11.583 \\
                    11.2 & 20.9264 & 20.468 \\
                    13.6 & 31.9084 & 31.476 \\
                    \hline
                \end{tabular}
            \end{table}
            
        
\section{実験の考察およびまとめ}
    \subsection{太陽電池の特性実験}
        \subsubsection{開放電圧および短絡電流の照度依存性試験}
            実験から、照度と開放電圧、短絡電流の間に明確な依存関係が存在することが確認されました。具体的には、開放電圧に関しては、半導体が持つ特定の閾値照度を超えると、その電圧は急激に上昇し、18V程度に安定する特性が観測されました。この挙動は半導体の特性として、一定のエネルギー(照度)を超えると電子が励起されやすくなり、それに伴い電圧が急激に上昇することを示していると考えられます。そして、その電圧が一定の値で安定するのは、励起される電子の数が飽和するためと推測されます。

            一方、短絡電流に関しては、照度の増加に伴い、直線的に増加する傾向が確認されました。これは、照度が高まるにつれて、半導体に入射する光子の数が増加し、それに伴って励起される電子の数が増えることを示唆しています。その結果、電流が比例的に増加することが確認されたのです。
        \subsubsection{電流電圧(I-V) 特性の実験}
            15Vまでの電圧値では、電流はほとんど変化していないとのことです。これは、太陽電池がその電圧範囲内で定電流モードで動作していることを示唆しています。しかし、15Vを超えると急激に電流が減少する現象は、太陽電池の飽和領域に達したことを意味しています。
        
            電力が特定の抵抗値まで上昇し、その後減少することが観測されたということは、太陽電池が最大電力点(MPP: Maximum Power Point)を持っていることを示しています。MPPは太陽電池が最大の出力を提供できる点を指します。
        \subsubsection{太陽電池の直列および並列接続の実験}
            太陽電池を並列接続すると、各太陽電池の電流が合算されるため、電流がおおよそ2倍に増加します。しかし、並列接続の際、各太陽電池からの電圧はそのまま出力となるので、電圧の変化は見られません。一方、太陽電池を直列接続すると、各太陽電池の電圧が合算される結果、出力電圧が倍増します。しかし、直列接続された回路内では流れる電流は一定のため、電流は太陽電池の数に関わらず変わらないことが確認されます。
        
    \subsection{風力発電の特性実験}
        \subsubsection{風速と回転性能試験}
            風力発電は風の運動エネルギーを電気エネルギーに変換する技術で、風速の増加は電圧と電流に線形の影響を与えます。しかし、電力と風速の関係は特定のXnの形を取ることが実験で確認され、これは風のエネルギーが風速の3乗に比例する特性を反映していると考えられます。
        \subsubsection{風速と発電特性の実験}
            実験の結果から、風力発電の特性において、特定の負荷抵抗で発電力が最大となる点、すなわちMaximum Power Point(MPP)が存在することが確認されました。このMPPは風の強さによって変動するため、風力発電の効率を最大化するには、この点を継続的に追跡し最適化する必要があることが示唆されます。
        \subsubsection{風速と充電特性の実験}
            風力発電の特性実験では、バッテリーの充電が開始するまでの電圧が低いこと、発電電流と充電電流、発電電力と充電電力の特性がほぼ一致していることが確認されました。これは、風力発電システムの効率が高く、生成された電力の大部分が充電に利用されていることを示しています。


    \subsection{事前課題}
        太陽電池では、光エネルギーが電気エネルギーに変換されます。これは、p型とn型の半導体が接合された場所で起こり、太陽光によって電子が励起され、起電力が生じます。
        一方、風力発電の発電量は風速の3乗に比例し、ローターの半径の2乗にも影響されます。風速が2倍になると発電量は8倍になりますが、実際の発電量は風車の設計や効率によっても変わります。
    \subsection{まとめ}
        太陽電池は、光起電力効果を用いて電力を生成し、特に単結晶と多結晶シリコン型を本実験で使用しました。一方、風力発電は風の運動エネルギーを電気に変換する技術で、風速の3乗に比例してエネルギーが生成されることが確認されました。さらに、風力発電の詳細実験では、風速と発電量の関係や、最大発電力点(MPP)の存在、および電力の効率的な充電特性が確認されました。
\end{document}
