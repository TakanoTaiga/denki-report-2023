% ドキュメントの設定
\documentclass[a4paper,11pt,xelatex,ja=standard]{bxjsarticle}
\usepackage{tikz}
\usetikzlibrary {datavisualization.formats.functions}
\usepackage{pgfplots}

% ドキュメント開始
\begin{document}

\section{実験の目的}

私たちの生活は多様なエネルギー資源に依存しています。これらは一般的に「枯渇性エネルギー」(石炭、石油など)と「再生可能エネルギー」(太陽光、風力など)に大別されます。枯渇性エネルギーは使用に伴い減少し、再生可能エネルギーは使用しても資源が枯渇しないとされています。化石燃料の消費は二酸化炭素を発生させ、これは温室効果ガスとして地球温暖化を引き起こす可能性があります。したがって、持続可能なエネルギー利用と環境保護を両立させるためには、再生可能エネルギーの利用拡大が不可欠です。

\section{実験の理論または原理}
    \subsection{太陽電池の特性実験}
        太陽電池(またはフォトボルタイックセル、ソーラーセル)は、光起電力効果を活用して電力を生成するデバイスです。具体的には、シリコンなどの半導体材料に光を照射すると、材料内部で電荷が発生し、外部回路に接続することで電力を抽出することができます。太陽電池には様々なタイプが存在し、主要なものとして単結晶シリコン型、多結晶シリコン型、アモルファスシリコン型、および色素増感型があります。本実験では、単結晶および多結晶シリコン型の太陽電池を使用します。

        光起電力効果とは、光が半導体材料に入射すると、光のエネルギーが電子に転送され、これによって電子が励起されて電流が発生する現象を指します。太陽電池の作動はこの効果に基づいており、入射光に含まれるエネルギーが半導体内の電子を動かし、この電子の動き(電流)を利用して電力を得るのです。
    \subsection{風力発電の特性実験}
        風力発電は風の運動エネルギーを利用して電気エネルギーに変換する技術です。風のエネルギーは風車を回転させ、その回転エネルギーが発電機を動かし電力を生成します。風のエネルギーがどれだけ電力生成に寄与するかは、風速の3乗に比例します。これは、風の運動エネルギーが質量と速度の二乗に比例し、質量が風速に比例するため、結果的に風のエネルギーは風速の3乗に比例するからです。

        風車には主に2つのタイプがあり、水平軸型は風向に追尾する必要がありますが、大型化が可能です。一方、垂直軸型は風向に左右されないが、軸の長さと保持に課題があります。

\section{実験の回路図または接続図}
    \subsection{太陽電池の特性実験}
    \subsection{風力発電の特性実験}
\section{実験の作業順序}
    \subsection{太陽電池の特性実験}
        \subsubsection{開放電圧および短絡電流の照度依存性試験}
            使用機器:太陽電池特性実験装置、照度計

            実験手順:
            \begin{enumerate}
                \item 以下の設定に注意し、コンセントを電源に差し込みます。
                \item 太陽電池を引き抜き、照度計のセンサ部を太陽電池が置かれていた部分に挿入し、セレクトスイッチを「設定」に変更した後、スイッチをONにします。スライドトランスを調整して、照度が200 Lux になるように調整します。
                \item セレクトスイッチを「計測」に変更し、太陽電池を挿入してスイッチをONにします。そのときの発生電圧を記録します。
                \item 負荷抵抗を最小にして、太陽電池を挿入してスイッチをONにしたときの発電電流を記録します。
                \item 照度をパラメータにして繰り返します(200 Lux から 20,000 Lux まで)。
            \end{enumerate}

        \subsubsection{電流電圧(I-V)特性の実験}
            使用機器:太陽電池特性実験装置、照度計

            実験手順:
            \begin{enumerate}
                \item 太陽電池を引き抜き、照度計のセンサ部を太陽電池を置く部分に挿入し、セレクトスイッチを「設定」に変更した後、スイッチをONにします。スライドトランスを調整して、照度が20,000 Lux になるように調整します。
                \item 太陽電池を挿入し、セレクトスイッチを「計測」に変更します。そのとき、負荷抵抗を変化させたときの(100\% から 0\%)、発電電圧および発電電流を記録します。
                \item 上記を照度を2,000 Lux に変更して繰り返します。
            \end{enumerate}

        \subsubsection{太陽電池の直列および並列接続の実験}
            使用機器: 工作用太陽電池(2個)、照度計、デジタルマルチテスタ(2台)

            実験手順:
            \begin{enumerate}
                \item 太陽電池を引き抜き、照度計のセンサ部を太陽電池が置かれていた部分に挿入し、セレクトスイッチを「設定」に変更した後、スイッチをONにします。スライドトランスを調整して、照度が20,000 Lux になるように調整します。
                \item 工作用太陽電池(1つ)を受光面に置き、太陽電池に接続する抵抗を変化させながら、発電電圧と発電電流をデジタルマルチテスタで記録します。
                \item 工作用太陽電池を直列接続および並列接続にして、抵抗を変化させたときの発電電圧と発電電流をデジタルマルチテスタで記録します。
            \end{enumerate}


    \subsection{風力発電の特性実験}
        \subsubsection{風速と回転性能試験}
            結線: 制御盤の電圧・電流端子に電圧計および電流計を接続する。
    
            実験手順:
            \begin{enumerate}
                \item 電圧計・電流計を接続する。負荷スイッチを「負荷」にする。負荷抵抗を100 Ωにする。
                \item 「始動」ボタンを押す。
                \item 風速調整ボリュームを徐々に回し,周波数が5.0 Hzになるように設定したときの風速・回転数・発電電圧・消費電流を記録する。
                \item 周波数を5.0~50.0 Hzまで変更して,各種データを記録する。このとき,風車が回転する周波数・風速を記録すること。
                \item 風速調整ボリュームをゼロに戻し,「停止」ボタンを押した後,負荷抵抗を20 Ωに変更して,上記(2)~(5)の実験を繰り返す。
                \item 風速調整ボリュームをゼロに戻し,「停止」ボタンを押して停止させる。
            \end{enumerate}
    
        \subsubsection{風速と回転性能試験}
            実験手順:
            \begin{enumerate}
                \item 「始動」ボタンを押す。
                \item 風速調整ボリュームを徐々に回し,周波数が35 Hzになるように設定したときの風速・回転数・発電電圧・消費電流を記録する。
                \item 抵抗値を変化させて(180~10 Ω),各種データを記録する。
                \item 周波数を50 Hzに変更して,上記(3)・(4)の実験を繰り返す。
                \item 風速調整ボリュームをゼロに戻し,「停止」ボタンを押して停止させる。
            \end{enumerate}
    
        \subsubsection{風速と回転性能試験}
            結線: 制御盤内部アルミボックスに備え付けられた電圧・電流端子に電圧計および電流計を接続する。
    
            実験手順:
            \begin{enumerate}
                \item 負荷スイッチを「バッテリ」にする。
                \item 風速調整ボリュームを徐々に回し,周波数が5.0 Hzになるように設定したときの風速・回転数・発電電圧・消費電流・充電電圧・充電電流を記録する。
                \item 周波数を10.0~60.0 Hzまで変更して,各種データを記録する。このとき,風車が回転する周波数・風速を記録すること。
                \item 実験装置を停止させる。
                \item 周囲を清掃して実験を終了する。
            \end{enumerate}

\section{実験の結果}
    \subsection{太陽電池の特性実験}
        \subsubsection{開放電圧および短絡電流の照度依存性試験}

            このグラフは照度が0から約7,500 luxまでの範囲で増加すると、開放電圧も急速に増加することを示しています。照度が約2,500luxを超えると、開放電圧はほぼ一定の値(約18V)に安定します。
            \begin{figure}[htbp]
                \centering
                \begin{tikzpicture}[scale=0.9]
                    \datavisualization[ 
                        scientific axes,
                        visualize as line/.list={voltage}, 
                        voltage={style={thick,mark=*,black},label in legend={text=実験結果}},
                        legend={south east inside},
                        x axis={label={照度[lux]},length=10cm},
                        y axis={label={電圧[V]},ticks={step=2},length=7cm},
                    ]
                    data[set=voltage] {
                            x, y
                            7,	0.9
                            200,	16.7
                            500,	17.7
                            1000,	18.2
                            2000,	18.7
                            3500,	19
                            5000,	19.4
                            10000,	19.8
                            15000,	20.1
                            20000,	20.3
                    };
                \end{tikzpicture}
                \caption{開放電圧と照度の依存関係}
            \end{figure}

            グラフから、照度が増加するにつれて短絡電流も増加していることが確認できます。2,500luxから線形に推移していることがわかります。
            \begin{figure}[htbp]
                \centering
                \begin{tikzpicture}[scale=0.9]
                    \datavisualization[ 
                        scientific axes,
                        visualize as line/.list={amp}, 
                        amp={style={thick,mark=*,black},label in legend={text=実験結果}},
                        legend={south east inside},
                        x axis={label={照度[lux]},length=10cm},
                        y axis={label={電流[A]},ticks={step=0.05},length=7cm},
                    ]
                    data[set=amp] {
                            x, y
                            7,	0
                            200,	0.012
                            500,	0.029
                            1000,	0.05
                            2000,	0.081
                            3500,	0.128
                            5000,	0.171
                            10000,	0.252
                            15000,	0.362
                            20000,	0.463
                    };
                \end{tikzpicture}
                \caption{短絡電流と照度の依存関係}
            \end{figure}
        \subsubsection{電流電圧(I-V)特性の実験}
            \begin{figure}[htbp]
                \centering
                \begin{tikzpicture}[scale=0.9]
                    \datavisualization[ 
                        scientific axes,
                        visualize as line/.list={200lux, 20klux}, 
                        200lux={style={thick,mark=*,black},label in legend={text=200lux}},
                        20klux={style={thick,mark=triangle,black},label in legend={text=2万lux}},
                        legend={north west outside},
                        x axis={label={電圧[V]},length=10cm},
                        y axis={label={電流[A]},length=7cm},
                    ]
                    data[set=200lux] {
                        x, y
                        0,	    0.093
                        0.5,	0.095
                        3,	    0.094
                        4.5,	0.093
                        7.3,	0.093
                        8.9,	0.093
                        11.3,	0.093
                        13.4,	0.093
                        15.6,	0.092
                        16.3,	0.086
                        16.7,	0.082
                    }
                    data[set=20klux] {
                        x, y
                        1.4,	0.463
                        5.1,	0.462
                        8,	    0.454
                        14.6,	0.457
                        16.1,	0.443
                        18.2,	0.312
                        18.9,	0.237
                        19.2,	0.186
                        19.4,	0.15
                        19.5,	0.129
                        19.5,	0.113
                        19.6,	0.098
                        19.5,	0.095
                    };
                \end{tikzpicture}
                \caption{I-V特性グラフ}
            \end{figure}

            \begin{figure}[htbp]
                \centering
                \begin{tikzpicture}[scale=0.9]
                    \datavisualization[ 
                        scientific axes,
                        visualize as line/.list={200lux, 20klux}, 
                        200lux={style={thick,mark=*,black},label in legend={text=200lux}},
                        20klux={style={thick,mark=triangle,black},label in legend={text=2万lux}},
                        legend={north west outside},
                        x axis={label={抵抗[Ω]},length=10cm},
                        y axis={label={電力[W]},length=7cm},
                    ]
                    data[set=200lux] {
                        x, y
                        0,	0
                        10,	0.0475
                        20,	0.282
                        30,	0.4185
                        40,	0.6789
                        50,	0.8277
                        60,	1.0509
                        70,	1.2462
                        80,	1.4352
                        90,	1.4018
                        100,	1.3694
                    }
                    data[set=20klux] {
                        x, y
                        0,	0.6482
                        10,	2.3562
                        15,	3.632
                        20,	6.6722
                        25,	7.1323
                        30,	5.6784
                        40,	4.4793
                        50,	3.5712
                        60,	2.91
                        70,	2.5155
                        80,	2.2035
                        90,	1.9208
                        100, 1.8525
                    };
                \end{tikzpicture}
                \caption{P-R特性グラフ}
            \end{figure}
        \subsubsection{太陽電池の直列および並列接続の実験}
    \subsection{風力発電の特性実験}
\section{実験の考察およびまとめ}
\section{参考文献}

\end{document}