% ドキュメントの設定
\documentclass[a4paper,11pt,xelatex,ja=standard]{bxjsarticle}
\usepackage{tikz}
\usetikzlibrary {datavisualization.formats.functions}
\usepackage{pgfplots}

% ドキュメント開始
\begin{document}

\section{実験の目的}

我々の生活は多様なエネルギー資源に依存している。これらは一般に、「枯渇性エネルギー」(石炭、石油など)と「再生可能エネルギー」(太陽光、風力など)に大別される。枯渇性エネルギーは使用に伴い減少し、再生可能エネルギーは使用しても資源が枯渇しないとされている。化石燃料の消費は二酸化炭素を発生させ、これは温室効果ガスとして地球温暖化を引き起こす可能性がある。したがって、持続可能なエネルギー利用と環境保護を両立させるためには、再生可能エネルギーの利用拡大が不可欠である。

\section{実験の理論または原理}
    \subsection{太陽電池の特性実験}
        太陽電池(またはフォトボルタイックセル、ソーラーセル)は、光起電力効果を活用して電力を生成するデバイスである。具体的には、シリコン等の半導体材料に光を照射すると、材料内部で電荷が発生し、外部回路に接続することで電力を抽出することができる。太陽電池には様々なタイプが存在し、主要なものとして単結晶シリコン型、多結晶シリコン型、アモルファスシリコン型、および色素増感型がある。本実験では、単結晶および多結晶シリコン型の太陽電池を使用する。

        光起電力効果とは、光が半導体材料に入射すると、光のエネルギーが電子に転送され、これによって電子が励起されて電流が発生する現象を指す。太陽電池の作動はこの効果に基づいており、入射光に含まれるエネルギーが半導体内の電子を動かし、この電子の動き(電流)を利用して電力を得る。
    \subsection{風力発電の特性実験}
        風力発電は風の運動エネルギーを利用して電気エネルギーに変換する技術である。風のエネルギーは風車を回転させ、その回転エネルギーが発電機を動かし電力を生成する。風のエネルギーがどれだけ電力生成に寄与するかは、風速の3乗に比例する。これは、風の運動エネルギーが質量と速度の二乗に比例し、質量が風速に比例するため、結果的に風のエネルギーは風速の3乗に比例するからである。

        風車には主に2つのタイプがあり、水平軸型は風向に追尾する必要がありますが、大型化が可能である。一方、垂直軸型は風向に左右されないが、軸の長さと保持に課題がある。


\section{実験の回路図または接続図}
    \subsection{太陽電池の特性実験}
    \subsection{風力発電の特性実験}
\section{実験の作業順序}
    \subsection{太陽電池の特性実験}
        \subsubsection{開放電圧および短絡電流の照度依存性試験}
            使用機器:太陽電池特性実験装置、照度計

            実験手順:
            \begin{enumerate}
                \item 以下の設定に注意し、コンセントを電源に差し込む。
                \item 太陽電池を引き抜き、照度計のセンサ部を太陽電池が置かれていた部分に挿入し、セレクトスイッチを「設定」に変更したのち、スイッチをONにする。スライドトランスを調整して、照度が200 Lux になるようにする
                \item セレクトスイッチを「計測」に変更し、太陽電池を挿入してスイッチをONにする。そのときの発生電圧を記録する。
                \item 負荷抵抗を最小にして、太陽電池を挿入してスイッチをONにしたときの発電電流を記録する。
                \item 照度をパラメータにして繰り返す。( 200 Lux \textasciitilde 20,000 Lux )
            \end{enumerate}
        \subsubsection{電流電圧(I-V)特性の実験}
            使用機器:太陽電池特性実験装置、照度計

            実験手順:
            \begin{enumerate}
                \item 太陽電池を引き抜き、照度計のセンサ部を太陽電池を置く部分に挿入し、セレクトスイッチを「設定」に変更したのち、スイッチをONにする。スライドトランスを調整して,照度が20,000 Lux になるようにする。
                \item 太陽電池を挿入し、セレクトスイッチを「計測」に変更する。そのとき、負荷抵抗を変化させたときの(100\% \textasciitilde 0\%)、発電電圧および発電電流を記録する。
                \item 上記を照度を2,000 Lux に変更して繰り返す。
            \end{enumerate}

        \subsubsection{太陽電池の直並列接続の実験}
            使用機器:工作用太陽電池(2個)、照度計、デジタルマルチテスタ(2台)

            実験手順:

            \begin{enumerate}
                \item [照度の設定]太陽電池を引き抜き、照度計のセンサ部を太陽電池が置かれていた 部分に挿入し、セレクトスイッチを「設定」に変更したのち、スイッチをONにす る。スライドトランスを調整して、照度が 20,000 Lux になるようにする。
                \item 工作用太陽電池(1つ)を受光面に置き,太陽電池に接続する抵抗を変化させなが ら,発電電圧と発電電流をマルチデジタルテスタで記録する。
                \item 工作用太陽電池を直列接続および並列接続にして,抵抗を変化させたときの発電電圧と発電電流をマルチデジタルテスタで記録する。
            \end{enumerate}


    \subsection{風力発電の特性実験}



\section{実験の結果}
\section{実験の考察およびまとめ}
\section{参考文献}

\end{document}