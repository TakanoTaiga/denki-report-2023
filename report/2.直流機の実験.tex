% ドキュメントの設定
\documentclass[a4paper,11pt,xelatex,ja=standard]{bxjsarticle}
\usepackage{tikz}
\usetikzlibrary {datavisualization.formats.functions}
\usepackage{pgfplots}
\usepackage{float}

% ドキュメント開始
\begin{document}

\section{実験の目的}
    \begin{enumerate}
        \item 直流電動機・発電機の運転方法を習得する。
        \item 直流電動機における複数の速度制御方法について習得する。
        \item 講義で習得した諸特性が実験においても得られることを確認する。
    \end{enumerate}
\section{実験の理論または原理}
    テキストの通り
\section{実験の作業順序}
    下記の実験についてテキストに記載された方法で実験した。
    \begin{enumerate}
        \item 始動器による直流電動機の始動試験
        \item 界磁制御法による直流電動機の速度制御
        \item サイリスタレオナードによる直流電動機の速度制御
        \item 直流発電機の無負荷飽和特性試験(他励)
        \item 直流発電機の無負荷飽和特性試験(自励)
        \item 直流発電機の負荷特性試験
        \item 直流電動機の負荷特性試験
    \end{enumerate}
\section{実験の結果}
    \subsection{始動器による直流電動機の始動試験}
        分巻電動機では、界磁コイルに直列に接続されている。複巻電動機では、界磁コイルと電機子コイルが直列と並列二つ接続されている。そのような理由から複巻電動機の方が直列インピーダンスが小さいため始動電流が小さくなる。
        そのような理由から分巻より複巻の方が電流が小さくなることを確認できた。

        \begin{table}[ht]
            \centering
            \begin{tabular}{|c|c|}
            \hline
            複巻 & 電流[A] \\
            \hline
             & 29.5 \\
            \hline
            \end{tabular}
            \quad
            \begin{tabular}{|c|c|}
            \hline
            分巻 & 電流[A] \\
            \hline
             & 31.5 \\
            \hline
            \end{tabular}
            \end{table}

    \subsection{界磁制御法による直流電動機の速度制御}
        界磁電流を増加させると、磁場の強さが増加しモーターの逆起電力が増加し、同じ供給電圧で流れる電流が減少し回転数が落ちる。
        以下の実験結果から上記のことが確認できた。
        \begin{figure}[H]
            \centering
            \begin{tikzpicture}[scale=0.9]
                \datavisualization[ 
                    scientific axes,
                    visualize as scatter,
                    scatter={style={mark=*, black}},
                    x axis={label={界磁電流 Ifm [A]},length=10cm},
                    y axis={label={回転速度 nm [rpm]},length=6cm,min value=0, max value=1600},
                ]
                data {
                    x, y
                    1.56, 1320.0
                    1.22, 1412.4
                    1.52, 1330.8
                    1.46, 1349.6
                    1.42, 1365.6
                    1.33, 1389.2
                    1.20, 1426.8
                    1.10, 1460.4
                    1.06, 1482.8
                    1.00, 1505.6
                };
            \end{tikzpicture}
            \caption{界磁電流と回転速度の関係}
        \end{figure}

        \begin{table}[h]
            \centering
            \begin{tabular}{|c|c|}
            \hline
            \textbf{$I_{fm}$[A]} & \textbf{$n_m$[rpm]} \\ \hline
            1.56 & 1320.0 \\ \hline
            1.22 & 1412.4 \\ \hline
            1.52 & 1330.8 \\ \hline
            1.46 & 1349.6 \\ \hline
            1.42 & 1365.6 \\ \hline
            1.33 & 1389.2 \\ \hline
            1.20 & 1426.8 \\ \hline
            1.10 & 1460.4 \\ \hline
            1.06 & 1482.8 \\ \hline
            1.00 & 1505.6 \\ \hline
            \end{tabular}
            \caption{Ifmとrpmの値}
            \label{tab:my_label}
        \end{table}
            

    \subsection{サイリスタレオナードによる直流電動機の速度制御}
        電圧が高くなると、電気し電機子電流が増加し、それに応じて発生するトルクが大きくなり。これにより、電動機はより高速で回転する。
        以下の実験結果から上記のことが確認できた。
        \begin{figure}[H]
            \centering
            \begin{tikzpicture}[scale=0.9]
                \datavisualization[ 
                    scientific axes,
                    visualize as scatter,
                    scatter={style={mark=*, black}},
                    x axis={label={$V$ [V]}, length=10cm, min value=0, max value=110},
                    y axis={label={$n_m$[rpm]}, length=6cm, min value=0, max value=1600},
                ]
                data {
                    x, y
                    100.0, 1495.3
                    91.0, 1356.5
                    78.4, 1171.4
                    70.4, 1046.1
                    60.4, 897.6
                    51.2, 754.4
                    40.8, 597.4
                    30.8, 450.9
                    21.4, 308.5
                    10.2, 145.9
                };
            \end{tikzpicture}
            \caption{V-rpm 特性グラフ}
        \end{figure}
    
        \begin{table}[ht]
            \centering
            \caption{電圧と回転数}
            \begin{tabular}{|c|c|}
            \hline
            $V$ [V] & $n_m$[rpm] \\
            \hline
            100.0 & 1495.3 \\\hline
            91.0 & 1356.5 \\\hline
            78.4 & 1171.4 \\\hline
            70.4 & 1046.1 \\\hline
            60.4 & 897.6 \\\hline
            51.2 & 754.4 \\\hline
            40.8 & 597.4 \\\hline
            30.8 & 450.9 \\\hline
            21.4 & 308.5 \\\hline
            10.2 & 145.9 \\\hline
            \end{tabular}
        \end{table}

    \subsection{直流発電機の無負荷飽和特性試験(他励)}
        直流発電機の特徴である出力電圧が上がるにつれ出力電流が増加するものと残留磁束により上がりより下りの方が電流が多いことが以下の実験結果から確認できた。
        \begin{figure}[H]
            \centering
            \begin{tikzpicture}[scale=0.9]
                \datavisualization[ 
                    scientific axes,
                    visualize as line/.list={index_a, index_b}, 
                    index_a={style={thick,mark=*,black},label in legend={text=Ifg 増加 Vg 増加}},
                    index_b={style={thick,dashed,mark=triangle,black},label in legend={text=Ifm 減少 Vg 減少}},
                    legend={north west outside},
                    x axis={label={電圧[V]},length=10cm},
                    y axis={label={電流[A]},length=6cm,min value=0, max value=130},
                ]
                data[set=index_a] {
                    x, y
                    0.00, 7
                    0.17, 21
                    0.24, 30
                    0.32, 39.8
                    0.39, 49.8
                    0.47, 60
                    0.56, 70
                    0.66, 80.2
                    0.76, 90
                    0.88, 100
                    1.03, 110.2
                    1.27, 120
                }
                data[set=index_b] {
                    x, y
                    0.00, 8
                    0.17, 20.2
                    0.24, 30
                    0.32, 39.8
                    0.39, 50
                    0.47, 57
                    0.56, 70.2
                    0.66, 79.8
                    0.76, 89.8
                    0.88, 100.2
                    1.03, 110
                };
            \end{tikzpicture}
            \caption{I-V特性グラフ}
        \end{figure}

    \subsection{直流発電機の無負荷飽和特性試験(自励)}
        直流発電機の特徴である出力電圧が上がるにつれ出力電流が増加するものと残留磁束により上がりより下りの方が電流が多いことが以下の実験結果から確認できた。

        \begin{figure}[H]
            \centering
            \begin{tikzpicture}[scale=0.9]
                \datavisualization[ 
                    scientific axes,
                    visualize as line/.list={index_a, index_b}, 
                    index_a={style={thick,mark=*,black},label in legend={text=増加}},
                    index_b={style={thick,dashed,mark=triangle,black},label in legend={text=減少}},
                    legend={north west outside},
                    x axis={label={電圧[V]},length=10cm},
                    y axis={label={電流[A]},length=6cm,min value=0, max value=120},
                ]
                data[set=index_a] {
                    x, y
                    0.00, 0
                    0.17, 7
                    0.24, 21
                    0.32, 30
                    0.39, 39.8
                    0.47, 49.8
                    0.56, 60
                    0.66, 70
                    0.76, 80.2
                    0.88, 90
                    1.03, 100
                    1.27, 110.2
                }
                data[set=index_b] {
                    x, y
                    0.00, 0
                    0.09, 8
                    0.16, 20.2
                    0.23, 30
                    0.31, 39.8
                    0.35, 50
                    0.47, 57
                    0.56, 70.2
                    0.65, 79.8
                    0.78, 89.8
                    0.96, 100.2
                };
            \end{tikzpicture}
            \caption{I-V特性グラフ}
        \end{figure}

    \subsection{直流発電機の負荷特性試験}
        負荷が小さくし電気子電流を増大させると出力電圧が下がることを確認できた。
        \begin{figure}[H]
            \centering
            \begin{tikzpicture}[scale=0.9]
                \datavisualization[ 
                    scientific axes,
                    visualize as scatter,
                    scatter={style={mark=*, black}},
                    x axis={label={電気子電流}, length=10cm, min value=0, max value=25},
                    y axis={label={電圧 [V]}, length=6cm, min value=0, max value=115}
                ]
                data {
                    x, y
                    4.04, 112.2
                    20.1, 100
                    23.4, 98
                    18.2, 104.2
                    15.17, 106
                    13.59, 106.4
                    11.43, 108.2
                    9.83, 108.8
                    7.71, 110.2
                    5.75, 112
                    1.96, 114
                    0, 114.2
                };
            \end{tikzpicture}
            \caption{電気子電流-電圧 特性グラフ}
        \end{figure}

        \begin{figure}[H]
            \centering
            \begin{tikzpicture}[scale=0.9]
                \datavisualization[ 
                    scientific axes,
                    visualize as scatter,
                    scatter={style={mark=*, black}},
                    x axis={label={電気子電流}, length=10cm, min value=0, max value=30},
                    y axis={label={回転数}, length=6cm, min value=0, max value=1620}
                ]
                data {
                    x, y
                    29.5, 1507.6
                    26.3, 1525.6
                    22.8, 1540.2
                    19, 1560.9
                    14.53, 1565.8
                    11.37, 1585.3
                    7.63, 1613.8
                };
            \end{tikzpicture}
            \caption{電気子電流-回転数 特性グラフ}
        \end{figure}
\section{考察}
    \subsubsection*{(1) 複巻電動機と分巻電動機の始動電流の大きさの違いが,何に起因するか考察しなさい.}

    実験結果に記載済み

    \subsubsection*{(2) 今回の実験で行った 2 種類の速度制御方法について,それぞれの特徴をまとめ,どのような場合に適するかについて述べなさい.}

    サイリスタレオナード制御は高精度な速度制御が可能で高速応答性を持ち、主に精密制御が必要な工業用途(例:工業用ロボット、CNCマシン)に適用されるのに対し、界磁制御はトルクを一定に保ちながら速度を変更することができ、より単純でコストが低く、重工業分野(例:電気機関車、起重機での使用に適している。

    \subsubsection*{(3) 実験4と5で得られたそれぞれの無負荷特性曲線の違いについて述べなさい.}

    電機子反作用などが存在しないため他励式発電機は界磁電流に対する端子電圧の変化は一定である。自励式発電機は電機子電流が界磁巻線を通るため特性曲線の変化が上昇と減少でことなる。

    \subsubsection*{(4) 実験6で得られた電圧変動率 $\varepsilon[\%]$ を求め,これについて考察しなさい.考察項目として,例えば,一般的な直流発電機の電圧変動率と比較するなどがある.}

    電圧変動率を求める式は

    $$
    \varepsilon[\%] = \frac{V_{max} - V_{min}}{V_{min}} \times 100
    $$

    上記の式に当てはめると$\varepsilon[\%] = 16.53 \%$である。一般的な発電機は$10\%$前後であることから高いと思われる。

    \subsubsection*{(5) 現在の電力システムは交流システムである。では,直流発電機および直流電動機がどのような用途で用いられているのかを調査しなさい.}

    現代における直流発電機の用途は限られているが、主に小規模発電や特定の工業用途で使用されている。例として自転車のダイナモがある。また、過去には電車などで利用されていたが、現在では交流システムが一般的になっている。

\end{document}