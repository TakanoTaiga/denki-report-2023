% ドキュメントの設定
\documentclass[a4paper,11pt,xelatex,ja=standard]{bxjsarticle}
\usepackage{tikz}
\usetikzlibrary {datavisualization.formats.functions}
\usepackage{pgfplots}
\usepackage{float}

% ドキュメント開始
\begin{document}

\section{目的}
    \begin{itemize}
        \item 単相変圧器
            \begin{itemize}
                \item 単相変圧器の極性試験ができる。
                \item 単相変圧器の実負荷試験により負荷特性を算出できる。
                \item 単相変圧器の等価回路定数を算出できる。
                \item アナログおよびデジタル電力計を使うことができる。
            \end{itemize}
        \item 三相交流
            \begin{itemize}
                \item 三相電力を測定できる(二電力計)。
                \item 単相変圧器の三相結線ができる。
            \end{itemize}
    \end{itemize}
\section{結果と考察}
    \subsection{単相変圧器の極性試験}
    以下の表の通り変圧器の極性は減極性だった。
        \begin{table}[H]
            \begin{tabular}{|c|c|c|c|}
                \hline
                V3[V] & V1[V] & V2[V] & 極性の判定 \\
                \hline
                97.5 & 200 & 102.5 & 減極性 \\
                \hline
            \end{tabular}
            \caption{極性}
        \end{table}
    \subsection{単相変圧器の実負荷試験}
        入力電圧が一定に固定しているため出力電圧はほぼ変化しなかった。しかし緩やか電圧が下がっているのは電流が増加したことで巻線などで$I^2R$の影響が大きくなっていると思われる。

        \begin{table}[H]
            \centering
            \begin{tabular}{|c|c|c|c|c|c|c|c|}
            \hline
            1次電圧[V] & 1次電流[A] & 入力電力[W] & 2次電圧[V] & 2次電流[A] & 出力電力[W] & 効率[\%] \\
            \hline\hline
            201 & 10.4 & 2089 & 100 & 20.0 & 2004 & 95.9 \\
            201 & 0.3  & 28   & 103 & 0    & 0    & 0.0  \\
            201 & 1.2  & 232  & 103 & 1.9  & 204  & 87.9 \\
            201 & 2.2  & 441  & 104 & 4.0  & 411  & 93.2 \\
            201 & 3.2  & 649  & 102 & 6.0  & 616  & 94.9 \\
            201 & 4.4  & 880  & 102 & 8.2  & 842  & 95.7 \\
            201 & 5.2  & 1057 & 102 & 9.9  & 1015 & 96.0 \\
            201 & 6.2  & 1258 & 102 & 11.9 & 1211 & 96.3 \\
            201 & 7.4  & 1486 & 101 & 14.1 & 1411 & 95.0 \\
            201 & 8.4  & 1683 & 101 & 16.0 & 1619 & 96.2 \\
            201 & 9.4  & 1894 & 101 & 18.1 & 1821 & 96.1 \\
            201 & 10.4 & 2086 & 100 & 20   & 2000 & 95.9 \\
            201 & 11.4 & 2290 & 199 & 21.9 & 2193 & 95.8 \\
            \hline
            \end{tabular}
        \end{table}

        \begin{figure}[H]
            \centering
            \begin{tikzpicture}[scale=0.9]
                \datavisualization[ 
                    scientific axes,
                    visualize as line/.list={index_a}, 
                    index_a={style={thick,mark=*,black},label in legend={text=電圧[V]}},
                    x axis={label={電流[A]},length=10cm},
                    y axis={label={電圧[V]},length=6cm,min value=0, max value=125},
                ]
                data[set=index_a] {
                    x, y
                    20.02, 100.2
                    0, 103.3
                    1.98, 102.9
                    4.00, 102.6
                    6.02, 102.3
                    8.26, 101.9
                    9.99, 101.6
                    11.91, 101.5
                    14.18, 101
                    16.09, 100.7
                    18.12, 100.6
                    20, 100
                    21.98, 99.8
                };
            \end{tikzpicture}
            \caption{I-V特性グラフ}
        \end{figure}

        \begin{figure}[H]
            \centering
            \begin{tikzpicture}[scale=0.9]
                \begin{axis}[
                    xlabel={電流[A]},
                    ylabel={効率[\%]},
                    xmin=0, xmax=22,
                    ymin=0, ymax=100,
                    legend style={at={(0.03,0.97)},anchor=south west},
                    grid=both,
                    minor tick num=1,
                    width=10cm,
                    height=6cm,
                ]
                
                \addplot[mark=*,mark size=2pt,mark options={color=black},thick] coordinates {
                    
                    (0, 0)
                    (1.98, 87.9)
                    (4.00, 93.2)
                    (6.02, 94.9)
                    (8.26, 95.7)
                    (9.99, 96.0)
                    (11.9, 96.3)
                    (14.18, 95.0)
                    (16.09, 96.2)
                    (18.12, 96.1)
                    (20, 95.9)
                    (20.02, 95.9)
                    (21.98, 95.8)
                };
                
                \legend{データ}
                \end{axis}
            \end{tikzpicture}
            \caption{I-効率特性グラフ}
        \end{figure}

        \subsection{単相変圧器の極性試験}
            \begin{table}[H]
                \centering
                \begin{tabular}{|c|c|c|}
                \hline
                電圧@一次[V] & 電圧@二次[V] & 電圧変動率[\%] \\
                \hline
                100.2 & 103.3 & 3.09 \\
                \hline
                \end{tabular}
                \caption{電圧データ}
            \end{table}

            \subsection{無負荷特性試験}
                \begin{table}[H]
                    \centering
                    \begin{tabular}{|c|c|c|c|c|c|c|}
                    \hline
                    一次電圧[V] & 一次電流[A] & 入力電力[W] & Yo[S] & go[S] & bo[S] \\
                    \hline
                    200.1 & 0.374 & 27 & 0.00187 & 0.135 & $0.134i$ \\
                    \hline
                    \end{tabular}
                    \caption{3.1のデータ}
                \end{table}

            \subsection{短絡試験}
                \begin{table}[H]
                    \centering
                    \begin{tabular}{|c|c|c|c|c|c|c|}
                    \hline
                    一次電圧[V] & 一次電流[A] & 入力電力[W] & Zo[Ω] & R[Ω] & X[Ω] \\
                    \hline
                    7.7 & 9.851 & 61 & 0.782 & 0.629 & 0.46 \\
                    \hline
                    \end{tabular}
                \caption{3.2のデータ}
                \end{table}
        \subsection{三相電力の測定}
            \begin{table}[h]
                \centering
                \caption{電力計データ}
                \begin{tabular}{|c|c|c|c|c|c|c|}
                    \hline
                    & 電力計1 & 電力計2 & 有効電力 & 無効電力 & 皮相電力 & 力率 \\
                    \hline
                    1 & 1200 & 1180 & 2380 & 34.6 & 2380.3 & 1.00 \\
                    0.8 & 570 & 1438 & 2008 & 1503.4 & 2508.4 & 0.80 \\
                    0.3 & 1166 & 78 & 1244 & 1884.4 & 2258.0 & 0.55 \\
                    \hline
            \end{tabular}

        \begin{table}[H]
        \centering
        \begin{tabular}{|c|c|c|c|c|}
        \hline
        1次線間電圧 & 1次相電圧 & 2次線間電圧 & 2次相電圧 & 結線方法 \\
        \hline
        200 & 114 & 58.8 & 57.8 & Y-$\Delta$ \\
        \hline
        200 & 199 & 101.1 & 100.5 & $\Delta$-$\Delta$ \\
        \hline
        200 & 199.7 & 174 & 100.5 & $\Delta$-Y \\
        \hline
        \end{tabular}
        \caption{電圧データと結線方法の表}
        \end{table}

        \begin{table}[H]
            \centering
            \begin{tabular}{|c|c|c|c|c|c|c|}
            \hline
            側 & 線間電圧 (V) & 相電圧 (V) & 線電流 (A) & 相電流 (A) & 電力 (kW) & 力率 \\
            \hline
            1次側 & 196.1 & - & 4.994 & 2.84 & 1.685 & 0.993 \\
            2次側 & 173.2 & 199.7 & 5.355 & - & 1.593 & 1 \\
            \hline
            \end{tabular}
            \caption{電力変換装置の性能データ}
            \label{table:performance_data}
            \end{table}
            
\end{table}

\clearpage
            
\section{考察実験の手順・結果・考察}

    デルタ結線と同様の手順で結線を行い電源を一つだけ取り除く。

    その結果

    $Pv = 0.527[kw]$

    $V=100[v]$

    $I-3.04[A]$

    $\frac{Pv}{2P}=\frac{0.527*10^3}{2*100*3.04} = 0.866$

\section{まとめ}

この実験を通して変圧器の基本的な特性から内部の電圧降下などを確認できた。

\end{document}